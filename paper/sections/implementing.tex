\section{Implementing the editor calculus}
\label{sec:implementing}

The implementation of the editor calculus requires
%
\begin{itemize}
  \item implementing the semantics of the calculus and allowing for an
        appropriate visualization of editing
  \item implementing a type checker based on the type system
\end{itemize}
%
Perhaps surprisingly, it also involves finding a way to script editor
calculus expressions. In this section we outline this.

\subsection{Implementing the editor calculus semantics}

We model the formation rules of the editor calculus in Elm as a collection of
algebraic datatypes. As an example, expressions are given by the
datatype

\begin{lstlisting}[language=elm,%
           label={edt-without-holes-definitions},%
           gobble=4,%
           caption={Formation rules (\ref{edt-formation-rules}) modeled in Elm},%
           ]
    type Edt
        = Pre Aep Edt
        | Bicond Eed Edt Edt
        | SeqComp Edt Edt
        | Rec Var.Id Edt
        | Call Var.Id
        | Nil
\end{lstlisting}
The rest of the formation rules are defined similarly.


\begin{figure*}
  \center
  \renewcommand{\arraystretch}{2}
  \begin{tabular}{llllll}
    \scriptsize(AT-VAR)      & $\AtVar$      & \scriptsize(AT-CONST)  & $\AtConst$   & \scriptsize(AT-HOLE)  & $\AtHole$  \\
    \scriptsize(AT-APP)      & $\AtApp$      & \scriptsize(AT-ABS)    & $\AtAbs$     & \scriptsize(AT-BREAK) & $\AtBreak$ \\
    \scriptsize(POS-TRIVIAL) & $\PosTrivial$                                                                              % & \scriptsize(POS-ABS)   & $\PosAbs$    & \scriptsize(POS-BREAK) & $\PosBreak$ \\
    \scriptsize(POS-APP-1)   & $\PosAppOne$  & \scriptsize(POS-APP-2) & $\PosAppTwo$                                      % & \scriptsize(NEC-TRIVIAL) & $\NecTrivial$ \\
    %\scriptsize(NEC-APP)     & $\NecApp$     & \scriptsize(NEC-ABS)   & $\NecAbs$    & \scriptsize(NEC-BREAK)   & $\NecBreak$
  \end{tabular}
  \caption{Selected condition reduction rules}
  \label{fig:conditionreductionrules}
\end{figure*}

The transition relation is implemented using the \elm{eval} function (from the
\elm{Editor.Edt} module) which computes transitions.

The implementation of checking if conditions in the spatial logic are
satisfied requires special attention. The checks for the three
modalities are implemented in the functions $\elm{isEquivalent}$,
$\elm{possibly}$ and $\elm{necessarily}$, and checks the current AST
under the cursor.

For the $\elm{isEquivalent}$ function the
\hyperref[fig:conditionreductionrules]{(AT-...)} rules are implemented by a
pattern match over the parameters. The function applies the rules very
straightforwardly such that if one of the clauses holds then it returns
$\elm{True}$ and if not then it returns $\elm{False}$. From the rule
\hyperref[fig:conditionreductionrules]{(AT-CONST)} it was not clear whether the
constants have to be equal or just some constant in order for $\at(\const c)$
to hold. We chose that the constants must be equal.

The $\elm{possibly}$ function is reduced by first checking whether we have the
trivial case \hyperref[fig:conditionreductionrules]{(POS-TRIVIAL)}. If we do
not, we check whether any of the other rules holds, i.e. if we are at an
application, abstraction or breakpoint, it calls recursively on the children of
the AST node. It deviates from the rules in that it combine the rules
\hyperref[fig:conditionreductionrules]{(POS-APP-1) and (POS-APP-2)} by a logical
or-operator.

Finally, we have added a notion of history not present in the
semantics of \cite{10.1145/3441296.3441393} that allows the programmer
to undo and redo individual edit actions. 

\subsection{Showing abstract syntax trees}

Our editor visualizes abstract syntax trees in order to make
structural editing intuitive for a user.  We are not necessarily
interested in finding a ``best'' visualization, nor in implementing a
vast number of different representations. The current version allows
for two distinct visualizations.

A simple representation is to represent the AST in a textual form,
as seen in figure \ref{fig:ast-in-text-form}.

\begin{figure}[H]
  \Large
  \begin{equation*}
    \cursor{(\app{\breakpoint{(\lambda{x}{(\app{x}{x})})}}
      {(\lambda{x}{(\app{x}{x})})})}
  \end{equation*}
  \caption{An AST visualized in textual form.}
  \label{fig:ast-in-text-form}
\end{figure}

The notation used for this representation follows that of Godiksen
et al.~\pepm. It is straightforward but can also be
difficult to reason about, especially as the AST grows large.

We also implement another representation, namely as a tree as seen in
figure \ref{fig:ast_visual_tree} where each construct is represented as a node.

Switching between the two visualizations is immediate in the user interface.

Each representation has a term in the $\elm{Representation}$ type and has a
function that computes the view. Thus, extending the editor with additional
representations is possible by extending the $\elm{Representation}$ type and
implementing a new visualization function.

\begin{figure*}
  \center
  \noindent\begin{minipage}{.45\textwidth}
    \center
    \includegraphics[width=\textwidth]{assets/ast_root_cursor.png}
  \end{minipage}\hfill
  \begin{minipage}{.45\textwidth}
    \center
    \includegraphics[width=\textwidth]{assets/ast_subtree_cursor.png}
  \end{minipage}\hfill
  \caption{An AST before and after cursor movement visualized in tree form}
  \label{fig:ast_visual_tree}
\end{figure*}

\subsection{Implementing a type checker}

Our editor type checks configurations $\breakpoint{E,a}$. We say that
$\breakpoint{E,a}$ is well-typed iff both the AST $a$ is
well-formed and well-typed in the empty AST context and the editor
expression $E$
is both well-formed and well-typed for the path to the cursor in the AST and
the editor context containing all valid paths in the AST \pepm. This
is captured by the type rule
(\ref{eq:welltypedconf}) where $\emptyset$ is the empty AST context, $p$ is the
path from the root to the cursor in $a$, and $\ctx{e}$ is the editor context
containing some pair for each valid path in $a$.
%
\begin{equation}\label{eq:welltypedconf}
  \TConfiguration
\end{equation}
%
We have implemented the AST type rules using a function that recursively
pattern matches each node of the given AST top-down. In the AST type rules we
judge a non-leaf AST node relative to the type of its children. In order to
model this relation, we use the \elm{Maybe}-type in the return type of the
function such that \elm{Nothing} represents an ill-typed AST and \elm{Just
$\tau$} represents a well-typed AST with type $\tau$. E.g. for abstractions we
follow the rule \hyperref[fig:asttyperules]{(T-LAMBDA)} straightforwardly by
judging the type of its subtree and then wrapping it in the abstraction type,
remembering to add the binding to the context.

\begin{lstlisting}[language=elm,%
                    gobble=8,%
                    mathescape,%
                    ]
        Lambda x tau a_ ->
            Maybe.map (TLambda tau) <|
                typed a_ <| Ctx.bind x tau actx
\end{lstlisting}

Editor expression type judgements are a bit more involved. Since
editor expressions get their type information from an AST and not
themselves, we use Bool as return type rather than Maybe as we did for
AST rules. We require that the editor expression is completed. As we
did for AST type judgements, we pattern match on a given editor
expression and type check on a case by case basis.  E.g. regarding
substitution with abstractions, seen in the rule
\hyperref[fig:edttyperules]{(T-SUB-ABS)}, we first check if the path $p$ is
valid in the editor type context $\ctx{e}$ and check for type consistency
between the bound type at path $p$ and the type of the given expression to be
substituted. Then we find the new context $\ctx{e}'$ by adding two new bindings
to the editor context found by restricting $\ctx{e}$ to only include bindings
in $\bar{p}$. Penultimately we check for type consistency between the bound
type at path $p$ and the type of the given expression to be substituted. Lastly
we recursively type check the next editor expression $E$ in the context
$\ctx{e}'$.
\begin{lstlisting}[language=elm,%
                    gobble=8,%
                    mathescape,%
                    ]
        Pre (Aep.Sub
                 (Aam.Lambda x tau1 tau2)) e ->
            case Ctx.get p ectx of
               Just (actx, atyp) -> let ectx_ =
                 Ctx.bind (Pth.extend p Ast.One)
                 (Ctx.bind x tau1 actx, tau2) <|
                   Ctx.bind p
                   (actx, Ast.TLambda tau1 tau2)
                     <| restrictBarPath p ectx
                 in
                 Ast.isConsistent atyp
                   (Ast.TLambda tau1 tau2)
                 && typed p ectx_ e
               Nothing -> False
\end{lstlisting}

\subsection{Building editor expressions}

Editor expressions written in a text field would have to be lexed and
parsed.  This would then introduce syntax errors to the editor but
with respect to editor expressions, and this goes against the spirit
of the editor. We have therefore created a way to build editor
expressions in a similar manner to ASTs.

In order to build them, we need to introduce holes to the editor
expressions. We define a hole term constructor for each type of editor
expression, as seen in figure \ref{fig:editorexpressionswithholes}

\begin{figure}[H]
  \hspace{-6mm}
  \begin{tabular}{p{3.9cm}p{3.9cm}}
    \begin{lstlisting}[language=elm,%
                            gobble=8,%
                            mathescape,%
                            ]
             type Aep
                = Eval
                    $\vdots$
                | AepHole
        \end{lstlisting} &

    \begin{lstlisting}[language=elm,%
                            gobble=8,%
                            mathescape,%
                            ]
            type Eed
                = Neg Eed
                     $\vdots$
                | EedHole
        \end{lstlisting} \\

    \begin{lstlisting}[language=elm,%
                            gobble=8,%
                            mathescape,%
                            ]
            type Edt
                = Pre Aep Edt
                    $\vdots$
                | EdtHole
        \end{lstlisting} &

    \begin{lstlisting}[language=elm,%
                            gobble=8,%
                            mathescape,%
                            ]
            type Aam
                = Var Var.Id
                    $\vdots$
                | AamHole
        \end{lstlisting}
  \end{tabular}
  \caption{Editor expression definitions with holes}
  \label{fig:editorexpressionswithholes}
\end{figure}

We are now able to build editor expressions by initializing the editor
expression builder with an \texttt{EdtHole}, and then allow the user to
substitute holes with appropriate expressions. As with ASTs, we have the concept
of atomic editor expressions, which are editor expressions with holes as
children. The user can substitute holes until no holes are left. An editor
expression without any holes is said to be \textit{completed}.

We are only interested in evaluating completed editor expressions. To do this,
we introduce a type variable to each editor expression type. The type variable
is used for recursively defining the type in terms of the type variable, but
also as an argument to each hole constructor, as seen in the following
listing.

\begin{lstlisting}[language=elm,%
                   label="eed-definitions",%
                   gobble=4,%
                   ]
    type Eed a
        = Neg (Eed a)
        | Conjunction (Eed a) (Eed a)
        | Disjunction (Eed a) (Eed a)
        | At (Aam a)
        | Possibly (Aam a)
        | Necessarily (Aam a)
        | EedHole a
\end{lstlisting}

We utilize the \texttt{()} and the \texttt{Never} types to represent
completed and uncompleted editor expressions. If for example \texttt{a} in
\texttt{Edt a} is replaced with \texttt{()}, the editor expression can contain
holes. Conversely, if we replace \texttt{a} with \texttt{Never}, it cannot
contain holes, since we need a \texttt{Never} value to construct a hole. We can
therefore use \texttt{Edt ()} to represent uncompleted editor expressions, and
\texttt{Edt Never} to represent completed editor expressions. We have created a
type alias for completed and uncompleted editor expressions of every type.

This approach creates strong guarantees by making impossible states impossible.
The \texttt{toCompleted} function cannot take an \texttt{uncompleted} editor
expression, return a \texttt{completed} editor expression and still have bugs,
since we have constrained the types to create type level guarantees.

\subsection{The user interface}
\label{user-interface}

The user interface allows for two different visualizations of programs: a
textual representation and a tree representation. The implementation of the
textual representation is simple: A conversion of the AST data structure to
HTML, as in Figure \ref{fig:ast-in-text-form}, but with syntax highlighting.

The implementation of the tree representation is more involved. Figure
\ref{fig:final_ui} shows an example of this.

The visualization of the editor expression builder is similar to the textual
representation of ASTs. We implemented a function that given an \texttt{Edt}
returns HTML. An example of an
editor expression being built is in the upper left corner of the images in
figure \ref{fig:final_ui}.

\begin{figure*}
  \begin{center}
    \includegraphics[width=0.7\textwidth]{assets/final_ui1.png}
  \end{center}
  \caption{A tree visualization}
  \label{fig:final_ui}
\end{figure*}

%%% Local Variables:
%%% mode: latex
%%% TeX-master: "../pepm2023"
%%% End:


