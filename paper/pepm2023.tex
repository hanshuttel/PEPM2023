\documentclass[sigplan,screen]{acmart} 
%
\usepackage[utf8]{inputenc}
\usepackage[british]{babel}
\usepackage{hyperref}
%\usepackage[left=0.5in, right=0.5in, top=1in, bottom=1in]{geometry}
\usepackage{graphicx}
\usepackage{listings}
\usepackage{xcolor}
\usepackage{amsmath,amssymb,amsthm}
\usepackage{fancyhdr}
\usepackage{extramarks}
\usepackage{enumerate}
\usepackage{subcaption}
\usepackage{caption}
\usepackage{float}
\usepackage{stmaryrd}
\usepackage{amstext}
\usepackage{mathtools}
\usepackage{multicol}
\usepackage{url}
\usepackage{hyperref}
\usepackage{array}

% References
\newcommand\pepm{\cite{10.1145/3441296.3441393}}
\newcommand\cornell{\cite{10.1145/358746.358755}}
\newcommand\hazel{\cite{conf/popl/Hazelnut17}}
\newcommand\scratch{\cite{10.1145/1592761.1592779}}
\newcommand\alice{\cite{10.1145/332040.332481}}
\newcommand\elmcore{\cite{Elm-lang-core}}
\newcommand\elmsvg{\cite{Elm-lang-svg}}
\newcommand\elmmat{\cite{Elm-lang-material}}
\newcommand\elmhtml{\cite{Elm-lang-html}}

% Metadata
\newcommand\mdtitle{Implementation of a Type-Safe Structure Editor}
\newcommand\mdtitleshort{Implementation of a Type-Safe Structure Editor}
\newcommand\mdauthor{
  Nicolaj Richs-Jensen \\
  \url{vzq239@alumni.ku.dk}
  \and
  Thorbjørn Bülow Bringgaard \\
  \url{vwc415@alumni.ku.dk}
  \and
  Tórur Feilberg Zachariasen \\
  \url{xsz482@alumni.ku.dk}
}
\newcommand\mdauthorshort{Nicolaj, Thorbjørn, Tórur \& Hans}
\newcommand\mddate{\today}

% Syntax
\newcommand\app[2]{#1~#2}
\newcommand\abs[2]{\lambda #1.#2}
\newcommand\cursor[1]{\llbracket #1 \rrbracket}
\newcommand\breakpoint[1]{\langle #1 \rangle}
\newcommand{\conf}[1]{\langle #1 \rangle}
\newcommand\hole{\llparenthesis\rrparenthesis}
\newcommand\transition[1]{\xrightarrow[]{\text{#1}}}

\newcommand\letin[3]{\texttt{let}~#1~#2~#3}
\newcommand\letrec[3]{\texttt{letrec}~#1~#2~#3}
\newcommand\match[1]{\texttt{match}~#1}

\newcommand\AST{\textbf{Ast}}
\newcommand\Ast{%
  a &::= x
  \mid c
  \mid \app{a_1}{a_2}
  \mid \abs{x}{a}
  \mid \cursor{a}
  \mid \breakpoint{a}
  \mid \hole
}

% Cursor contexts
\newcommand{\cursorhole}{\[\cdot\]}
\newcommand{\Cts}{C &::= [\cdot] \mid \app{C}{\hat{a}} \mid \app{\hat{a}}{C} \mid \abs{x}{C} \mid \breakpoint{C}}

\newcommand\aHat{%
  \hat{a} &::= x
  \mid c
  \mid \app{\hat{a}_1}{\hat{a}_2}
  \mid \abs{x}{\hat{a}}
  \mid \breakpoint{\hat{a}}
  \mid \hole
}

\newcommand\aDot{%
  \dot{a} &::= \cursor{\hat{a}}
  \mid \abs{x}{\cursor{\hat{a}}}
  \mid \app{\cursor{\hat{a}_1}}{\hat{a}_2}
  \mid \app{\hat{a}_1}{\cursor{\hat{a}_2}}
  \mid \breakpoint{\cursor{\hat{a}}}
}

% Editor expressions
\newcommand\pre[2]{#1.#2}
\newcommand\bicond[3]{\ensuremath{#1 \Rightarrow #2 \mid #3}}
\newcommand\seqcomp[2]{\ensuremath{#1 \ggg #2}}
\newcommand\rec[2]{\ensuremath{\texttt{rec}~#1.#2}}
\newcommand\call[1]{#1}
\newcommand\nil{\mathbf{0}}
\newcommand\Exp[2]{\breakpoint{#1,\, #2}}

\newcommand\edt{\textbf{Edt}}
\newcommand\Edt{%
  E &::= \pre{\pi}{E}
  \mid \bicond{\phi}{E_1}{E_2}
  \mid \seqcomp{E_1}{E_2}
  \mid \rec{x}{E}
  \mid \call{x}
  \mid \nil
}

% Prefix commands
\newcommand\eval{\texttt{eval}}
\newcommand\sub[1]{\ensuremath{\{ #1 \}}}
\newcommand\child[1]{\ensuremath{\texttt{child}~#1}}
\newcommand\parent{\texttt{parent}}

\newcommand\aep{\textbf{Aep}}
\newcommand\Aep{%
  \pi &::= \eval
  \mid \sub{D}
  \mid \child{n}
  \mid \parent
}

% AST node modifiers
\newcommand\var[1]{\texttt{var}~x}
\newcommand\const[1]{\texttt{const}~c}
\newcommand\aamApp{\texttt{app}}
\newcommand\aamLambda[1]{\texttt{lambda}~#1}
\newcommand\aamBreak{\texttt{break}}
\newcommand\aamHole{\texttt{hole}}

\newcommand\aam{\textbf{Aam}}
\newcommand\Aam{%
  D &::= \var{x}
  \mid \const{c}
  \mid \aamApp
  \mid \aamLambda{x}
  \mid \aamBreak
  \mid \aamHole
}

% Conditions
\newcommand\conjunction[2]{#1 \land #2}
\newcommand\disjunction[2]{#1 \lor #2}
\newcommand\at[1]{\ensuremath{@#1}}
\newcommand\possibly[1]{\ensuremath{\Diamond #1}}
\newcommand\necessarily[1]{\ensuremath{\Box #1}}

\newcommand\eed{\textbf{Eed}}
\newcommand\Eed{%
  \phi &::= \neg\phi
  \mid \conjunction{\phi_1}{\phi_2}
  \mid \disjunction{\phi_1}{\phi_2}
  \mid \at{D}
  \mid \possibly{D}
  \mid \necessarily{D}
}

% Reduction rules
%% Table 1
\newcommand\CondOne{%
  \frac{a \vDash \phi}
  {\Exp{\bicond{\phi}{E_1}{E_2}}{a}\transition{$\epsilon$}\Exp{E_1}{a}}
}

\newcommand\CondTwo{%
  \frac{a \nvDash \phi}
  {\Exp{\bicond{\phi}{E_1}{E_2}}{a}\transition{$\epsilon$}\Exp{E_2}{a}}
}

\newcommand\Eval{%
  \frac{a \to v}
  {\Exp{\pre{\eval}{E}}{a}\transition{$v$}\Exp{E}{a}}
}

\newcommand\Seq{%
  \frac{\Exp{E_1}{a}\transition{$\alpha$}\Exp{E_1'}{a'}}
  {\Exp{\seqcomp{E_1}{E_2}}{a}\transition{$\alpha$}\Exp{\seqcomp{E_1'}{E_2}}{a'}}
}

\newcommand\Struct{%
  \frac{E_1 \equiv E_2\quad\Exp{E_2}{a}\transition{$\alpha$}\Exp{E_2'}{a'} \quad E_2' \equiv E_1'}
  {\Exp{E_1}{a}\transition{$\alpha$}\Exp{E_1'}{a'}}
}

\newcommand\Context{%
  \begin{aligned}
     & \frac{a \transition{$\pi$} a'}
    {\Exp{\pre{\pi}{E}}{C[a]}\transition{$\pi$}\Exp{E}{C[a']}} \\
     & \text{where } \pi \neq \eval
  \end{aligned}
}

%% Table 2
\newcommand\Var{%
  \frac{}
  {\cursor{\hat{a}}\transition{\{var x\}}\cursor{x}}
}

\newcommand\Hole{%
  \frac{}
  {\cursor{\hat{a}}\transition{\{hole\}}\cursor{\hole}}
}

\newcommand\Const{%
  \frac{}
  {\cursor{\hat{a}}\transition{\{const c\}}\cursor{c}}
}

\newcommand\App{%
  \frac{}
  {\cursor{\hat{a}}\transition{\{app\}}\cursor{\app{\hole}{\hole}}}
}

\newcommand\BreakOne{%
  \frac{\hat{a} \neq \breakpoint{\hat{a}'}}
  {\cursor{\hat{a}}\transition{\{break\}}\cursor{\breakpoint{\hat{a}}}}
}

\newcommand\BreakTwo{%
  \frac{}
  {\cursor{\breakpoint{\hat{a}}}\transition{\{break\}}\cursor{\hat{a}}}
}

\newcommand\LAMBDA{%
  \frac{}
  {\cursor{\hat{a}}\transition{\{lambda x\}}\cursor{\abs{x}{\hole}}}
}

%% Table 3
\newcommand\AppCOne{%
  \frac{}
  {\cursor{\app{\hat{a}_1}{\hat{a}_2}}\transition{child 1}\app{\cursor{\hat{a}_1}}{\hat{a}_2}}
}

\newcommand\AppCTwo{%
  \frac{}
  {\cursor{\app{\hat{a}_1}{\hat{a}_2}}\transition{child 2}\app{\hat{a}_1}{\cursor{\hat{a}_2}}}
}

\newcommand\AppPOne{%
  \frac{}
  {\app{\cursor{\hat{a}_1}}{\hat{a}_2}\transition{parent}\cursor{\app{\hat{a}_1}{\hat{a}_2}}}
}

\newcommand\AppPTwo{%
  \frac{}
  {\app{\hat{a}_1}{\cursor{\hat{a}_2}}\transition{parent}\cursor{\app{\hat{a}_1}{\hat{a}_2}}}
}

\newcommand\ABSC{%
  \frac{}
  {\cursor{\abs{x}{\hat{a}}}\transition{child 1}\abs{x}{\cursor{\hat{a}}}}
}

\newcommand\ABSP{%
  \frac{}
  {\abs{x}{\cursor{\hat{a}}}\transition{parent}\cursor{\abs{x}{\hat{a}}}}
}

\newcommand\BreakC{%
  \frac{}
  {\cursor{\breakpoint{\hat{a}}}\transition{child 1}\breakpoint{\cursor{\hat{a}}}}
}

\newcommand\BreakP{%
  \frac{}
  {\breakpoint{\cursor{\hat{a}}}\transition{parent}\cursor{\breakpoint{\hat{a}}}}
}

%% Table 4
\newcommand\AtVar{%
  \frac{}
  {x \vDash \at{(\text{var }y)}}
}

\newcommand\AtConst{%
  \frac{}
  {c \vDash \at{(\text{const }c)}}
}

\newcommand\AtHole{%
  \frac{}
  {\hole \vDash \at{\text{hole}}}
}

\newcommand\AtApp{%
  \frac{}
  {\app{\hat{a}_1}{\hat{a}_2} \vDash \at{\text{app}}}
}

\newcommand\AtAbs{%
  \frac{}
  {\abs{x}{\hat{a}} \vDash \at{\text{lambda }y}}
}

\newcommand\AtBreak{%
  \frac{}
  {\breakpoint{\hat{a}} \vDash \at{\text{break}}}
}

%% Table 5
\newcommand\PosTrivial{%
  \frac{\hat{a} \vDash \at{D}}
  {\hat{a} \vDash \possibly{D}}
}

\newcommand\PosAppOne{%
  \frac{\hat{a}_1 \vDash \possibly{D}}
  {\app{\hat{a}_1}{\hat{a}_2} \vDash \possibly{D}}
}

\newcommand\PosAppTwo{%
  \frac{\hat{a}_2 \vDash \possibly{D}}
  {\app{\hat{a}_1}{\hat{a}_2} \vDash \possibly{D}}
}

\newcommand\PosAbs{%
  \frac{\hat{a} \vDash \possibly{D}}
  {\abs{x}{\hat{a}} \vDash \possibly{D}}
}

\newcommand\PosBreak{%
  \frac{\hat{a} \vDash \possibly{D}}
  {\breakpoint{\hat{a}} \vDash \possibly{D}}
}

\newcommand\NecTrivial{%
  \frac{\hat{a} \vDash \at{D}}
  {\hat{a} \vDash \necessarily{D}}
}

\newcommand\NecApp{%
  \frac{\hat{a}_1 \vDash \possibly{D}\quad\hat{a}_2 \vDash \possibly{D}}
  {\app{\hat{a}_1}{\hat{a}_2} \vDash \necessarily{D}}
}

\newcommand\NecAbs{%
  \frac{\hat{a} \vDash \possibly{D}}
  {\abs{x}{\hat{a}} \vDash \necessarily{D}}
}

\newcommand\NecBreak{%
  \frac{\hat{a} \vDash \possibly{D}}
  {\breakpoint{\hat{a}} \vDash \necessarily{D}}
}


%% Table 6
\newcommand\BConst{%
  \frac{}
  {c \to c}
}

\newcommand\BAbs{%
  \frac{}
  {\abs{x}{a}\to\abs{x}{a}}
}

\newcommand\BCursor{%
  \frac{a \to v}
  {\cursor{a} \to v}
}

\newcommand\BApp{%
  \begin{aligned}
     & \frac{a_1 \to \abs{x}{a_1'} \quad a_2 \to v \quad a_1'\{v/x\} \to v'}
    {\app{a_1}{a_2} \to v'}                                                  \\
     & \text{where } v \notin \textbf{BVal} \cup \textbf{HVal}
  \end{aligned}
}

\newcommand\BAppBOne{%
  \frac{a_1 \to w}
  {\app{a_1}{a_2} \to \app{w}{a_2}}
}

\newcommand\BAppBTwo{%
  \frac{a_1 \to \abs{x}{a_1'} \quad a_2 \to w}
  {\app{a_1}{a_2} \to \app{\abs{x}{a_1'}}{w}}
}

\newcommand\BAppHOne{%
  \frac{a_1 \to u \quad a_2 \to v}
  {\app{a_1}{a_2} \to \app{u}{v}}
}

\newcommand\BAppHTwo{%
  \frac{a_1 \to \abs{x}{a_1'} \quad a_2 \to u}
  {\app{a_1}{a_2} \to \app{\abs{x}{a_1'}}{u}}
}

\newcommand\BBreakB{%
  \frac{}
  {\breakpoint{a}\to\breakpoint{a}}
}


% Values
\newcommand\val{\textbf{Val}}
\newcommand\Val{%
  v &::= c
  \mid \abs{x}{a}
  \mid w
  \mid u
}

\newcommand\hval{\textbf{HVal}}
\newcommand\HVal{%
  u &::= \hole
  \mid \app{u}{v}
  \mid \app{\abs{x}{a}}{u}
}

\newcommand\bval{\textbf{BVal}}
\newcommand\BVal{%
  w &::= \breakpoint{a}
  \mid \app{w}{a}
  \mid \app{\abs{x}{a}}{w}
}


% Types
\newcommand\pth{\textbf{Pth}}
\newcommand\Pth{p ::= p~T \mid \epsilon}
\newcommand\atyp{\textbf{ATyp}}
\newcommand\Atyp{%
  \tau ::= b
  \mid \tau_1 \to \tau_2
  \mid \breakpoint{\tau}
  \mid ~?
}
\newcommand\ctyp{\textbf{CTyp}}
\newcommand\Ctyp{T ::= \texttt{one} \mid \texttt{two}}
\newcommand\actx{\textbf{ACtx}}
\newcommand\ectx{\textbf{ECtx}}
\newcommand\elm[1]{\texttt{#1}}
\newcommand\typ[1]{\textbf{#1}}
\newcommand\ctx[1]{\Gamma_{#1}}
\newcommand\bind[2]{#1 : #2}

% Type rules
%% AST type rules
\newcommand\TConfiguration{%
  \frac{\emptyset \vdash \bind{a}{\tau} \quad p,\ctx{e} \vdash \bind{E}{ok}}
  {p,\ctx{e} \vdash \bind{\breakpoint{E,a}}{ok}}\\
}
\newcommand\TVar{%
  \frac{\ctx{a}(x) = \tau}
  {\ctx{a} \vdash \bind{x}{\tau}}
}

\newcommand\TConst{%
  \frac{}
  {\ctx{a} \vdash \bind{c}{b}}
}

\newcommand\TCursor{%
  \frac{\ctx{a} \vdash \bind{a}{\tau}}
  {\ctx{a} \vdash \bind{\cursor{a}}{\tau}}
}

\newcommand\TLambda{%
  \frac{\ctx{a},~\bind{x}{\tau_1} \vdash \bind{a}{\tau_2}}
  {\ctx{a} \vdash \abs{\bind{x}{\tau_1}}{\bind{a}{\tau_1\to\tau_2}}}
}

\newcommand\TBreak{%
  \frac{\ctx{a} \vdash \bind{a}{\tau}}
  {\ctx{a} \vdash \bind{\breakpoint{a}}{\breakpoint{\tau}}}
}

\newcommand\THole{%
  \frac{}
  {\ctx{a} \vdash \bind{(\bind{\hole}{\tau})}{\tau}}
}

\newcommand\TApp{%
  \frac{\ctx{a} \vdash \bind{a_1}{\tau_3} \quad \tau_3\sim\tau_1\to\tau_2 \quad \ctx{a} \vdash \bind{a_2}{\tau_1}}
  {\ctx{a} \vdash \bind{\app{a_1}{a_2}}{\tau_2}}
}

%% Edt type rules
\newcommand\TEval{%
  \frac{p,\ctx{e} \vdash \bind{E}{ok}}
  {p,\ctx{e} \vdash \bind{\pre{\eval}{E}}{ok}}
}

\newcommand\TRef{%
  \frac{}
  {p,\ctx{e} \vdash \bind{x}{ok}}
}

\newcommand\TNil{%
  \frac{}
  {p,\ctx{e} \vdash \bind{\nil}{ok}}
}

\newcommand\TChOne{%
  \frac{\ctx{e}(p~\texttt{one}) = (\ctx{a}, \tau) \quad p~\texttt{one}, \ctx{e} \vdash \bind{E}{ok}}
  {p, \ctx{e} \vdash \bind{\pre{(\child 1)}{E}}{ok}}
}

\newcommand\TChTwo{%
  \frac{\ctx{e}(p~\texttt{two}) = (\ctx{a}, \tau) \quad p~\texttt{two}, \ctx{e} \vdash \bind{E}{ok}}
  {p, \ctx{e} \vdash \bind{\pre{(\child 2)}{E}}{ok}}
}

\newcommand\TParent{%
  \frac{\ctx{e}(p) = (\ctx{a}, \tau) \quad p, \ctx{e} \vdash \bind{E}{ok}}
  {p~T, \ctx{e} \vdash \bind{\pre{\parent}{E}}{ok}}
}

\newcommand\TSeq{%
  \begin{aligned}
    & \frac{p,p'~\ctx{e} \vdash \bind{E_1}{ok} \quad p',\bar{p'}~\ctx{e} \vdash \bind{E_2}{ok}}
    {p,\ctx{e} \vdash \bind{\seqcomp{E_1}{E_2}}{ok}} \\
    & \text{for } p' = path(p,E_1)
  \end{aligned}
}

\newcommand\TRec{%
  \begin{aligned}
  & \frac{\ctx{e}(p) = (\ctx{a},\tau) \quad p,(\emptyset,\bind{p}{(\emptyset,?)}) \vdash \bind{E}{ok}}
  {p, \ctx{e} \vdash \bind{\rec{x}{E}}{ok}} \\
  & \text{if } path(p,E)=p
  \end{aligned}
}

\newcommand\TCond{%
  \begin{aligned}
  & \frac{p, update(\ctx{e},\delta \cup \{ (p, (\ctx{a}, \tau))\}) \vdash \bind{E_1}{ok} \quad p, \ctx{e} \vdash \bind{E_2}{ok}}
  {p, \ctx{e} \vdash \bind{\bicond{\phi}{E_1}{E_2}}{ok}}\\
  & \text{if~~~~} path(p,E_1) = path(p,E_2)\\
  & \text{and } (\ctx{a},\tau) = types(\phi, (\emptyset,?))\\
  & \text{and } \delta = \cap_{D \in limits(\phi,\aam)} follows(D,p)
  \end{aligned}
}

\newcommand\TSubVar{%
  \frac{\ctx{e}(p) = (\ctx{a}, \tau) \quad \ctx{a}(x) = \tau' \quad \tau \sim \tau' \quad p,(\bar{p}~\ctx{e}, \bind{p}{(\ctx{a},\tau')}) \vdash \bind{E}{ok}}
  {p, \ctx{e} \vdash \bind{\pre{\{\var x\}}{E}}{ok}}
}

\newcommand\TSubHole{%
  \frac{\ctx{e}(p) = (\ctx{a},\tau') \quad \tau \sim \tau' \quad p,(\bar{p}~\ctx{e},\bind{p}{(\ctx{a},\tau)}) \vdash \bind{E}{ok}}
  {p, \ctx{e} \vdash \bind{\pre{\{\bind{\texttt{hole}}{\tau}\}}{E}}{ok}}
}

\newcommand\TSubConst{%
  \frac{\ctx{e}(p)=(\ctx{a},\tau) \quad \tau \sim b \quad p, (\bar{p}~\ctx{e},\bind{p}{(\ctx{a},b)}) \vdash \bind{E}{ok}}
  {p, \ctx{e} \vdash \bind{\pre{\{\const c\}}{E}}{ok}}
}

\newcommand\TSubApp{%
  \begin{aligned}
  & \frac{\ctx{e}(p) = (\ctx{a},\tau_2') \quad \tau_2 \sim \tau_2' \quad p,\ctx{e}' \vdash \bind{E}{ok}}
  {p, \ctx{e} \vdash \bind{\pre{\{\bind{\texttt{app}}{\tau_1\to\tau_2,\tau_1}\}}{E}}{ok}}\\
  & \text{where } \ctx{e}' = \bar{p}~\ctx{e},\bind{p}{(\ctx{a},\tau_2)},\bind{p~\texttt{one}}{(\ctx{a},\tau_1\to\tau_2)},\bind{p~\texttt{two}}{(\ctx{a},\tau_1)}
  \end{aligned}
}

\newcommand\TSubBreak{%
  \begin{aligned}
  & \frac{p,\ctx{e}' \vdash \bind{E}{ok}}
    {p, \ctx{e} \vdash \bind{\pre{\{\texttt{break}\}}{E}}{ok}}\\
  & \text{where } \ctx{e}' = toggle(p,\ctx{e})
  \end{aligned}
}

\newcommand\TSubAbs{%
  \begin{aligned}
  & \frac{\ctx{e}(p)=(\ctx{a},\tau_3)\quad\tau_3\sim\tau_1\to\tau_2\quad p,\ctx{e}'\vdash\bind{E}{ok}}
  {p,\ctx{e}\vdash\bind{\pre{\{\bind{\texttt{lambda }x}{\tau_1\to\tau_2}\}}{E}}{ok}}\\
  & \text{where }\ctx{e}'=\bar{p}~\ctx{e},\bind{p}{(\ctx{a},\tau_1\to\tau_2)},\bind{p~\texttt{one}}{((\ctx{a},\bind{x}{\tau_1}),\tau_2)}
  \end{aligned}
}


\lstdefinelanguage{elm}
{
  % List of keywords
  morekeywords={
    alias,
    as,
    case,
    else,
    exposing,
    if,
    import,
    in,
    let,
    module,
    of,
    port,
    then,
    type,
    where
  },
  sensitive=true, % Keywords are case-sensitive
  morecomment=[s]{\{-}{-\}}, % s is for start and end delimiter
  morecomment=[l]{--},
  morestring=[b]" % Defines that strings are enclosed in double quotes
}

%%%%%%%%%%%%%%%%%%%%%%%%%%%%%%%
%% Define Colors             %%
%% Colors from: elm.lang.org %%
%%%%%%%%%%%%%%%%%%%%%%%%%%%%%%%

\usepackage{color}
\definecolor{elm-orange}{RGB}{240,20,20}
\definecolor{elm-gray}{RGB}{149,149,138}
\definecolor{elm-blue}{RGB}{0,133,255}

%%% Local Variables:
%%% mode: latex
%%% TeX-master: "pepm2023"
%%% End:


\settopmatter{printfolios=true}

%%% If you see 'ACMUNKNOWN' in the 'setcopyright' statement below,
%%% please first submit your publishing-rights agreement with ACM (follow link on submission page).
%%% Then please update our instructions page and copy-and-paste the NEW commands into your article.
%%% Please contact us in case of questions; allow up to 10 min for the system to propagate the information.
%%%
%%% The following is specific to PEPM '23 and the paper
%%% 'TITLE'
%%% by 

\setcopyright{acmcopyright}
\acmPrice{15.00}
\acmDOI{???}
\acmYear{2023}
\copyrightyear{2023}
\acmSubmissionID{???}
\acmISBN{978-1-4503-8305-9/21/01}
\acmConference[PEPM '23]{Proceedings of the 2023 ACM SIGPLAN Workshop
  on Partial Evaluation and Program Manipulation}{January
  2023}{Boston, USA}
\acmBooktitle{Proceedings of the 2023 ACM SIGPLAN Workshop
  on Partial Evaluation and Program Manipulation, Boston, USA}


\begin{document}
%\pagestyle{empty}
\pagenumbering{gobble}

%
\title{Implementing A Type-Safe Structure Editor}
%
%\titlerunning{Abbreviated paper title}
% If the paper title is too long for the running head, you can set
% an abbreviated paper title here
%

% \inst kan måske fjernes eftersom vi alle er fra samme institut.

\author[Hüttel]{Hans Hüttel}
\email{hans.huttel@di.ku.dk}
\orcid{0000-0002-4603-5407}

\affiliation{%
  \institution{Department of Computer Science, University of Copenhagen}
  \streetaddress{Selma Lagerlöfs Vej 300}
  \city{2100 København}
  \country{Denmark}}
  
\author[Richs-Jensen]{Nicolaj Richs-Jensen}
\email{vzq239@alumni.ku.dk}

\affiliation{%
  \institution{Department of Computer Science, University of Copenhagen}
  \streetaddress{Selma Lagerlöfs Vej 300}
  \city{2100 København}
  \country{Denmark}}

\author[Bringgaard]{Thorbjørn Bülow Bringgaard}
\email{vwc415@alumni.ku.dk}

\affiliation{%
  \institution{Department of Computer Science, University of Copenhagen}
  \streetaddress{Selma Lagerlöfs Vej 300}
  \city{2100 København}
  \country{Denmark}}

\author[Zachariassen]{Tórur Feilberg Zachariassen}
\email{xsz482@alumni.ku.dk}

\affiliation{%
  \institution{Department of Computer Science, University of Copenhagen}
  \streetaddress{Selma Lagerlöfs Vej 300}
  \city{2100 København}
  \country{Denmark}}
  
\keywords{Structure editors, type systems, functional programming.}

%
% First names are abbreviated in the running head.
% If there are more than two authors, 'et al.' is used.
%
%

%
 \begin{abstract}
   Syntax-directed editors provide an alternative to text-based
   editors. This paper describes the implementation in Elm of a
   type-safe calculus for constructing, editing and visualizing
   functional programs. The type system guarantees that only
   well-typed programs can be built. The implementation lets the
   programmer build edit expressions using a specialized editor and
   allows for the evaluation of incomplete programs. A type checker
   for the editor calculus guarantees that only well-typed edit
   expressions can be built, and a program visualizer allows the
   programmer to choose between different visualizations of code and
   how it is edited and evaluated.
\end{abstract}

\begin{CCSXML}
<ccs2012>
<concept>
<concept_id>10011007.10011006.10011039.10011311</concept_id>
<concept_desc>Software and its engineering~Semantics</concept_desc>
<concept_significance>500</concept_significance>
</concept>
<concept>
<concept_id>10003752.10010124.10010125.10010130</concept_id>
<concept_desc>Theory of computation~Type structures</concept_desc>
<concept_significance>500</concept_significance>
</concept>
<concept>
<concept_id>10011007.10011006.10011050.10011017</concept_id>
<concept_desc>Software and its engineering~Domain specific languages</concept_desc>
<concept_significance>300</concept_significance>
</concept>
</ccs2012>
\end{CCSXML}

\ccsdesc[500]{Software and its engineering~Semantics}
\ccsdesc[500]{Theory of computation~Type structures}
\ccsdesc[300]{Software and its engineering~Domain specific languages}

\maketitle              % typeset the header of the contribution

\thispagestyle{empty}

\section{Introduction}
\label{introduction}

In their 2020 paper Godiksen et al. \pepm~described an editor calculus for a
typed structure editor that works directly with abstract syntax trees (ASTs)
making syntax errors impossible. Their editor calculus has structural
operational semantics and works for a simple lambda calculus that allows
evaluation of unfinished programs through the notion of holes and breakpoints.
The calculus uses editor expressions to modify ASTs and has bidirectional
communication between editor expressions and ASTs such that an editor
expression modifies an AST and an AST provides information that the editor
expression needs in order to reduce. Our project aims to work as a proof of
concept for the calculus.\\

There are a multitude of reasons as to why structure editors can work as an
alternative to the standard text editors. This includes the elimination of
syntax errors by working directly with ASTs and in other instances giving a
more intuitive visualization of the program.

An early example of such a structure editor is the Cornell Program Synthesizer
from 1981 \cornell. It directly edits the ASTs in programs by using a cursor to
select nodes and allows for insertions and modifications at the cursor. This
editor was not typed meaning that it allowed ill-typed ASTs to be built.

A later example is Hazelnut from 2017 \hazel. It is a bidirectionally typed
structure editor on ASTs, where it introduces holes that represent uncompleted
subtrees. It has dynamic semantics and type consistency of holes are not
checked until the hole is built to completeness. The result is that for
uncompleted programs the completed parts can still be evaluated and the holes
can still be meaningful, even when they are not well-typed. In comparison, the
type-safe editor calculus described by Godiksen et al. has static semantics and
does assign meaning to ill-typed holes, but still allows partial evaluation of
programs through the insertion of breakpoints.

The visualization of the Cornell Program Synthesizer and Hazelnut
implementations are both textual and close to that of a text editor. Some
structure editors that are motivated by unconventional (non text-based)
visualization include Scratch \scratch, where programming is done by moving
blocks of different color and combining blocks like a puzzle. Another inclusion
is Alice \alice, which is a programming language for writing 3D graphics where
the structure editor is visualized as 3D renderings. \\

In this paper we will introduce the implementation details of our structure
editor derived from the editor calculus described in the paper by Godiksen et
al. and argue for why our implementation retains all guarantees from the
calculus, even where the calculus may not be directly modeled.

We want to implement the editor in a functional programming language, as they
provide expressive types that lend themselves well to modeling the formal
descriptions of the editor. In particular, we want a typed language that has
algebraic data types. We considered using either Haskell or Elm, since both are
fairly popular functional programming languages. Haskell would be a good
choice, but we ended up choosing Elm, primarily because it would be easier to
create a user interface in Elm.\\

In order to create editor expressions that are used to traverse and modify the
ASTs, we introduce an editor expression builder that should allow us to
iteratively build editor expressions.

We want to visualize ASTs and editor expressions as well as interact with them.
By interacting we mean to be able to modify an AST by evaluating configurations
consisting of an AST and an editor expression, where the editor expression can
be any desired expression. Furthermore we want choices available in our visual
representation of ASTs, making it possible to change between different
visualizations.

Godiksen et al. also describe a way to acquire type-safety in their calculus,
by way of their type system. The implementation of their type system should
make it impossible to get runtime errors from problems such as invalid ASTs as
well as expressions acting upon non-existing subtrees et cetera.

Going from a completely theoretical calculus to a practical implementation, we
also have to consider the usability of the editor. While not the main focus we
will go through approaches where we make it easier to use certain aspects of
the editor, such as using editor expressions and navigating the ASTs. \\

This paper is structured the following way. Preliminaries are listed in section
\ref{preliminaries}. The overall principles of Godiksen et al.s editor calculus
are described in section \ref{principles}. In section \ref{design} we introduce
the design of the implementation. Section \ref{modeling} is about how we model
the calculus terms, types and the editor expression builder. The evaluation
rules are defined in section \ref{evaluation} together with implementation and
execution of said rules. In section \ref{type-checking} we define the type
rules and describe the implementation of the type checker. We discuss the user
interface and how it combines the calculus in section \ref{user-interface}. In
the last section - section \ref{conclusion} - we conclude our project and
discuss future work.


\section{Programs}

The program terms that editor expression terms can build and modify are terms of a simply typed $\lambda$-calculus extended with breakpoints and holes. The formation rules are 

\begin{align}
  D & ::= \var{x}
  \mid \const{c}
  \mid \bind{\aamApp}{\tau_1\to\tau_2,\tau_1}  \\ \label{eq:lan-mod-aam}
&  \mid \bind{\aamLambda{x}}{\tau_1\to\tau_2} 
  \mid \aamBreak
  \mid \bind{\aamHole}{\tau} \\ 
  a & ::= x
  \mid c
  \mid \app{a_1}{a_2}
  \mid \abs{\bind{x}{\tau}}{a}
  \mid \cursor{a}
  \mid \breakpoint{a}
  \mid \bind{\hole}{\tau} \label{eq:lan-mod-ast}
\end{align}

The new constructions are $\bind{\aamHole}{\tau}$ which denotes a hole annotated with type $\tau$, $\cursor{a}$ which denotes an expression that is directly underneath the cursor and $\breakpoint{a}$, which is a breakpoint -- meaning that this occurrence of $a$ is left unevaluated.

%%% Local Variables:
%%% mode: latex
%%% TeX-master: "../pepm2023"
%%% End:

\section{The editor calculus}

\subsection{Syntax}

Editor expressions are given by the following formation rules.
%
\begin{align}
    \pi &::= \eval
  \mid \sub{D}
  \mid \child{n}
  \mid \parent \label{aep-formation-rules} \\
   \phi &::= \neg\phi
  \mid \conjunction{\phi_1}{\phi_2}
  \mid \disjunction{\phi_1}{\phi_2}
  \mid \at{D}
  \mid \possibly{D}
  \mid \necessarily{D} \label{eed-formation-rules} \\
    E &::= \pre{\pi}{E}
  \mid \bicond{\phi}{E_1}{E_2}
  \mid \seqcomp{E_1}{E_2}
  \mid \rec{x}{E}
  \mid \call{x}
  \mid \nil \label{edt-formation-rules} \\
    D &::= \var{x}
  \mid \const{c}
  \mid \aamApp
  \mid \aamLambda{x}
  \mid \aamBreak
  \mid \aamHole \label{aam-formation-rules}
\end{align}

Editor calculus expressions $E$ can examine or modify the subtree of the
AST that is currently pointed to by the cursor. Expressions are built
from atomic edit action prefixes $\pi$. The \sub{D} prefix substitutes
the content under the cursor by the term constructor $D$. Moreover,
conditional expressions \bicond{\phi}{E_1}{E_2} will proceed as $E_1$
if the condition $\phi$ holds for the current subtree and as $E_2$
otherwise. We allow for recursive expressions \rec{x}{E} where $x$
ranges over recursion variables (we only consider expressions where
every $x$ is bound by some \rec{x}{E}) and sequential composition
\seqcomp{E_1}{E_2}.

Conditions $\phi$ are conditions from a spatial logic on ASTs that contains
the usual propositional connectives as well as the modalities
$\at{\phi}$, $\possibly{\phi}$ and $\necessarily{\phi}$. The formula
$\at{\phi}$ expresses that $\phi$ holds for the current subtree, while
$\possibly{\phi}$ expresses that $\phi$ in some subtree of the current
subtree and $\necessarily{\phi}$ expresses that $\phi$ holds
everywhere in the current subtree.

Finally, the $\eval$ primitive allows us to evaluate the entire AST
(up to possible breakpoints).

\subsection{Semantics}

\begin{figure*}
  \center
  \renewcommand{\arraystretch}{2}
  \begin{tabular}{llll}
    \scriptsize(COND-1)  & $ \CondOne $           & \scriptsize(COND-2) & $ \CondTwo$ \\
    \scriptsize(EVAL)    & $ \Eval $              & \scriptsize(SEQ)    & $ \Seq$     \\
    \scriptsize(STRUCT)  & $\Struct$              & \scriptsize(CONTEXT)& \scriptsize$\Context$
  \end{tabular}
  \caption{Editor Expression reduction rules}
  \label{fig:edtreductionrules}
\end{figure*}

The semantics of the editor calculus is given by a labelled transition system
$(S_E, \mathcal{L}_E, \to)$, where $S_E = \edt \times \AST$ and $\mathcal{L}_E
= \val \cup \aep \cup \{\epsilon\}$ and where transitions take the form
$\breakpoint{E, a} \transition{$\alpha$} \breakpoint{E', a'}$ with $a$ being
well-formed, $E$ being completed and $\alpha \in \mathcal{L}_E$. 

Editor expressions are always reduced in configurations of the form
$\breakpoint{E,a}$, where $E$ is the given editor expression and $a$ is the
currently build AST to which $E$ is applied. In terms of the labeled
transition system, we incrementally reduce the configuration until $E'$ becomes
$\nil$, and returns $\alpha$ for each increment. Figure
\ref{fig:edtreductionrules} show the different reduction rules for editor
expressions.

As an example, the rules \hyperref[fig:edtreductionrules]{(COND-1) and (COND-2)} describe that
for a biconditional editor expression, $\bicond{\phi}{E_1}{E_2}$, we either use
$E_1$ or $E_2$ as editor expression for the next reduction increment, based on
whether or not $\phi$ holds for the AST in the configuration. Hence it just
returns $\epsilon$.

\subsection{A type system}

%%% Local Variables:
%%% mode: latex
%%% TeX-master: "../pepm2023"
%%% End:

\section{Implementing the editor calculus}
\label{sec:implementing}

The implementation of the editor calculus requires

\begin{itemize}
\item implementing the semantics of the calculus and allowing for an
  appropriate visualization of editing
\item implementing a type checker based on the type system
\end{itemize}

but perhaps surprisingly, it involves finding a way to script editor
calculus expressions. In this section we outline this.

\subsection{Implementing the editor calculus semantics}

We model the formation rules of the editor calculus by a collection of
algebraic datatypes. As an example, expressions are given by the
datatype

\begin{lstlisting}[language=elm,%
           label={edt-without-holes-definitions},%
           gobble=4,%
           caption={Formation rules (\ref{edt-formation-rules}) modeled in Elm},%
           ]
    type Edt
        = Pre Aep Edt
        | Bicond Eed Edt Edt
        | SeqComp Edt Edt
        | Rec Var.Id Edt
        | Call Var.Id
        | Nil
\end{lstlisting}

The rest of the formation rules are defined similarly.

\subsection{Abstract syntax trees}

We are interested in visualizing the AST to make the editor intuitive
for a user.  We are not necessarily interested in finding the ``best''
visualization, nor in implementing a vast number of different
representations.

A simple representation would be to represent the AST in a textual form,
as seen in figure \ref{fig:ast-in-text-form}.

\begin{figure}[H]
    \Large
    \begin{equation*}
      \cursor{(\app{\breakpoint{(\lambda{x}{(\app{x}{x})})}}
      {(\lambda{x}{(\app{x}{x})})})}
    \end{equation*}
    \caption{An AST visualized in textual form.}
    \label{fig:ast-in-text-form}
\end{figure}

The notation used for this representation follows that of Godiksen
et al.~\pepm. It is straightforward but can also be
difficult to reason about, especially as the AST grows large.

We also implement another representation, namely as a tree as seen in
figure \ref{fig:ast_visual_tree}.

Switching between the two visualizations is immediate in the user interface.

\begin{figure*}
  \center
  \noindent\begin{minipage}{.45\textwidth}
    \center
    \includegraphics[width=\textwidth]{assets/ast_root_cursor.png}
  \end{minipage}\hfill
  \begin{minipage}{.45\textwidth}
    \center
    \includegraphics[width=\textwidth]{assets/ast_subtree_cursor.png}
  \end{minipage}\hfill
  \caption{An AST before and after cursor movement visualized in tree form}
  \label{fig:ast_visual_tree}
\end{figure*}

\subsection{Implementing a type checker}

We type check with respect to a
configuration. We say that a configuration is well-typed iff; the AST is both
well-formed and well-typed in the empty AST context and the editor expression
is both well-formed and well-typed for the path to the cursor in the AST and
the editor context containing all valid paths in the AST \pepm. I.e. equation
(\ref{eq:welltypedconf}) where $\emptyset$ is the empty AST context, $p$ is the
path from the root to the cursor in $a$, and $\ctx{e}$ is the editor context
containing some pair for each valid path in $a$.

\begin{equation}\label{eq:welltypedconf}
  \TConfiguration
\end{equation}

\subsection{Building editor expressions}

To build editor expressions, we need to introduce holes to the editor
expressions. We define a hole term constructor for each type of editor
expression, as seen in figure \ref{fig:editorexpressionswithholes}

\begin{figure}[H]
  % \center
  \hspace{-6mm}
    \begin{tabular}{p{3.9cm}p{3.9cm}}
        \begin{lstlisting}[language=elm,%
                            gobble=8,%
                            mathescape,%
                            ]
             type Aep
                = Eval
                    $\vdots$
                | AepHole
        \end{lstlisting} &

        \begin{lstlisting}[language=elm,%
                            gobble=8,%
                            mathescape,%
                            ]
            type Eed
                = Neg Eed
                     $\vdots$
                | EedHole
        \end{lstlisting} \\

        \begin{lstlisting}[language=elm,%
                            gobble=8,%
                            mathescape,%
                            ]
            type Edt
                = Pre Aep Edt
                    $\vdots$
                | EdtHole
        \end{lstlisting} &

        \begin{lstlisting}[language=elm,%
                            gobble=8,%
                            mathescape,%
                            ]
            type Aam
                = Var Var.Id
                    $\vdots$
                | AamHole
        \end{lstlisting}
    \end{tabular}
    \caption{Editor expression definitions with holes}
    \label{fig:editorexpressionswithholes}
\end{figure}

We are now able to build editor expressions by initializing the editor
expression builder with an \texttt{EdtHole}, and then allow the user to
substitute holes with appropriate expressions. As with ASTs, we have the concept
of atomic editor expressions, which are editor expressions with holes as
children. The user can substitute holes until no holes are left. An editor
expression without any holes is said to be \textit{completed}.

We are only interested in evaluating completed editor expressions. Therefore, we
need to know when an editor expression is completed, and only then allow the
user to evaluate it. To do this, we introduce a type
variable to each editor expression type. The type variable is used
for recursively defining the type in terms of the type variable, but also as an
argument to each hole constructor, as seen in the following listings.

\begin{lstlisting}[language=elm,%
                   label="aep-definition",%
                   gobble=4,%
                   ]
    type Aep a
        = Eval
        | Sub (Aam a)
        | Child Ast.Child
        | Parent
        | AepHole a
\end{lstlisting}

\begin{lstlisting}[language=elm,%
                   label="eed-definitions",%
                   gobble=4,%
                   ]
    type Eed a
        = Neg (Eed a)
        | Conjunction (Eed a) (Eed a)
        | Disjunction (Eed a) (Eed a)
        | At (Aam a)
        | Possibly (Aam a)
        | Necessarily (Aam a)
        | EedHole a
\end{lstlisting}

\begin{lstlisting}[language=elm,%
                   label="generic-edt-definition",%
                   gobble=4,%
                   ]
    type Edt a
        = Pre (Aep a) (Edt a)
        | Bicond (Eed a) (Edt a) (Edt a)
        | SeqComp (Edt a) (Edt a)
        | Rec Var.Id (Edt a)
        | Call Var.Id
        | Nil
        | EdtHole a
\end{lstlisting}

\begin{lstlisting}[language=elm,%
                   label="aam-definitions",%
                   gobble=4,%
                   ]
    type Aam a
        = Var Var.Id
        | Con Const.Value
        | App ATyp ATyp
        | Lambda Var.Id ATyp ATyp
        | Break
        | Hole ATyp
        | AamHole a
\end{lstlisting}

We utilize the \texttt{()} and the \texttt{Never} types to represent
completed and uncompleted editor expressions. If for example \texttt{a} in
\texttt{Edt a} is replaced with \texttt{()}, the editor expression can contain
holes. Conversely, if we replace \texttt{a} with \texttt{Never}, it cannot
contain holes, since we need a \texttt{Never} value to construct a hole. We can
therefore use \texttt{Edt ()} to represent uncompleted editor expressions, and
\texttt{Edt Never} to represent completed editor expressions. We have created a
type alias for completed and uncompleted editor expressions of every type. For
example, the following are the type aliases for \texttt{Edt a}.

\begin{lstlisting}[language=elm,%
                   label="completed-and-uncompleted-edts",%
                   gobble=4,%
                   ]
    type alias Uncompleted =
        Edt ()

    type alias Completed =
        Edt Never
\end{lstlisting}

This approach creates stronger guarantees by making impossible states
impossible. The \texttt{toCompleted} function cannot take an
\texttt{uncompleted} editor expression, return a \texttt{completed} editor
expression and still have bugs, since we have constrained the types to create
type level guarantees.

\subsection{The user interface}
\label{user-interface}

The user interface allows for two different visualizations of
programs: a textual representation and a tree representation. The
implementation of the textual representation is simple: A conversion
of the AST data structure to HTML with syntax highlighting . The last
image in figure \ref{fig:final_ui} shows this.

The implementation of the tree representation is more involved. Figure
\ref{fig:ast_visual_tree} shows an example of this

The visualization of the editor expression builder is similar to the textual
representation of ASTs. We implemented a function that given an \texttt{Edt}
returns HTML. An example of an
editor expression being built is in the upper left corner of the images in
figure \ref{fig:final_ui}.

\begin{figure*}
  \begin{center}
      \includegraphics[width=0.7\textwidth]{assets/final_ui1.png}
 \end{center}
  \caption{A tree visualization}
  \label{fig:final_ui}
\end{figure*}

%%% Local Variables:
%%% mode: latex
%%% TeX-master: "../pepm2023"
%%% End:

\section{Conclusion}
\label{conclusion}

The intention of our project was to act as a form of proof of concept of the
editor calculus described by Godiksen et al. \pepm. We have described the
implementation details of our editor, which models the formal editor calculus
quite precisely. Part of the reason we could achieve this was our choice of
writing in Elm. We modeled the calculus terms precisely in Elm using algebraic
data types and the rules using pattern matches. Furthermore, we are able to
construct programs in our editor by extending ASTs iteratively with editor
expressions.

With our implemented editor expression builder, we have made it possible to
build editor expressions from scratch in such a way that only allows
syntactically valid editor expressions to be made and completed editor
expressions to be evaluated. Certain commonly used expressions are available as
predefined macros.

We partially implemented a type system that --- if fully implemented --- makes
reductions of ill-typed configurations, ill-typed ASTs and runtime errors
impossible. The unfinished implementations of the type system resulted in the
type system not being used in the final implementation, but the majority of the
system is implemented.

Lastly, we implemented visual representations of ASTs and editor expressions.
The representations of ASTs can be switched between textual and tree
representations. Furthermore, we made it possible to move the view around and
zoom in and out of the tree visualization.

\subsection{Future work}

The most crucial future work would be to fully implement the type system. In
particular to implement the last cases of \elm{Eed.types} and \elm{Edt.typed}.
As we do not have more details regarding this issue, we will focus on additional
features we had considered during development, that we never got around to.

\subsubsection{Collapsing subtrees}
For the purpose of making the tree-view more visually intuitive, one might want
to collapse subtrees. This comes with certain challenges. One problem is that if
a subtree can be collapsed, we likely want to ensure that the cursor is not in
that subtree, as the cursor would then be hidden. Similarly, if a subtree is
collapsed, the cursor should not be able to navigate into that subtree.

To address the latter problem, we can add additional data to every AST
term constructor that keeps track of whether the ``node'' is collapsed. With
this data, we can ensure that the function that navigates to a child node first
checks if the child is collapsed, and then navigates only if the child is not
collapsed.

The solution to the former problem depends on how we let the user collapse a
tree. One way would be to add a formation rule to editor expressions that
represents toggling the collapsed state of the node under the cursor. If this
approach is taken, we do not need to think about the cursor being in the subtree
that is being collapsed, since it per definition surrounds the subtree.

A second approach would be to let the user click on a node in the user
interface, and then toggle whether that node is collapsed or not. With this
approach we would have to check whether the cursor is inside the subtree, if it
is being collapsed. If the cursor is not in the subtree, we would simply
collapse the subtree, otherwise, we could either do nothing, or move the cursor
up each parent until it encapsulates the subtree to be collapsed.

Another consideration we have had for this feature, is that we could try and
evaluate the collapsed subtree. If this subtree is evaluated to a constant,
variable or some other short expression, we can show this as a summary instead
of showing the root of the subtree we have collapsed. \\
\subsubsection{View following cursor}
Another feature that would improve usability is to make the default of the view
to follow the cursor when moving through the tree. There are a few different
ways to do it. The editor can follow the cursor with it being centered in the
view. Alternatively it can have the view follow whenever the cursor is near the
edge. For either of these cases, the editor could have a button that moves the
focus back to the cursor, making it possible to toggle between the locked and
the free view.
\subsubsection{Saving and loading project}
Another feature that would be nice to have is being able to save and load a
project. This could for example be implemented as JSON encoders and decoders,
where the user could store to or load from a local JSON file.


%
% ---- Bibliography ----
%
% BibTeX users should specify bibliography style 'splncs04'.
% References will then be sorted and formatted in the correct style.
%

\bibliographystyle{ACM-Reference-Format}
\bibliography{references/references}



\end{document}

%%% Local Variables:
%%% mode: latex
%%% TeX-master: t
%%% End:
