\section{Conclusion}
\label{conclusion}

The intention of our project was to act as a form of proof of concept of the
editor calculus described by Godiksen et al. \pepm. We have described the
implementation details of our editor, which models the formal editor calculus
quite precisely. Part of the reason we could achieve this was our choice of
writing in Elm. We modeled the calculus terms precisely in Elm using algebraic
data types and the rules using pattern matches. Furthermore, we are able to
construct programs in our editor by extending ASTs iteratively with editor
expressions.

With our implemented editor expression builder, we have made it possible to
build editor expressions from scratch in such a way that only allows
syntactically valid editor expressions to be made and completed editor
expressions to be evaluated. Certain commonly used expressions are available as
predefined macros.

We partially implemented a type system that --- if fully implemented --- makes
reductions of ill-typed configurations, ill-typed ASTs and runtime errors
impossible. The unfinished implementations of the type system resulted in the
type system not being used in the final implementation, but the majority of the
system is implemented.

Lastly, we implemented visual representations of ASTs and editor expressions.
The representations of ASTs can be switched between textual and tree
representations. Furthermore, we made it possible to move the view around and
zoom in and out of the tree visualization.

\subsection{Future work}

The most crucial future work would be to fully implement the type system. In
particular to implement the last cases of \elm{Eed.types} and \elm{Edt.typed}.
As we do not have more details regarding this issue, we will focus on additional
features we had considered during development, that we never got around to.

\subsubsection{Collapsing subtrees}
For the purpose of making the tree-view more visually intuitive, one might want
to collapse subtrees. This comes with certain challenges. One problem is that if
a subtree can be collapsed, we likely want to ensure that the cursor is not in
that subtree, as the cursor would then be hidden. Similarly, if a subtree is
collapsed, the cursor should not be able to navigate into that subtree.

To address the latter problem, we can add additional data to every AST
term constructor that keeps track of whether the ``node'' is collapsed. With
this data, we can ensure that the function that navigates to a child node first
checks if the child is collapsed, and then navigates only if the child is not
collapsed.

The solution to the former problem depends on how we let the user collapse a
tree. One way would be to add a formation rule to editor expressions that
represents toggling the collapsed state of the node under the cursor. If this
approach is taken, we do not need to think about the cursor being in the subtree
that is being collapsed, since it per definition surrounds the subtree.

A second approach would be to let the user click on a node in the user
interface, and then toggle whether that node is collapsed or not. With this
approach we would have to check whether the cursor is inside the subtree, if it
is being collapsed. If the cursor is not in the subtree, we would simply
collapse the subtree, otherwise, we could either do nothing, or move the cursor
up each parent until it encapsulates the subtree to be collapsed.

Another consideration we have had for this feature, is that we could try and
evaluate the collapsed subtree. If this subtree is evaluated to a constant,
variable or some other short expression, we can show this as a summary instead
of showing the root of the subtree we have collapsed. \\
\subsubsection{View following cursor}
Another feature that would improve usability is to make the default of the view
to follow the cursor when moving through the tree. There are a few different
ways to do it. The editor can follow the cursor with it being centered in the
view. Alternatively it can have the view follow whenever the cursor is near the
edge. For either of these cases, the editor could have a button that moves the
focus back to the cursor, making it possible to toggle between the locked and
the free view.
\subsubsection{Saving and loading project}
Another feature that would be nice to have is being able to save and load a
project. This could for example be implemented as JSON encoders and decoders,
where the user could store to or load from a local JSON file.
