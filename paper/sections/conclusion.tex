\section{Conclusion}
\label{conclusion}

We have implemented the editor calculus described by Godiksen et
al. \pepm. The implementation includes a tool for writing editor
scripts and different visualizations of abstract syntax trees that
allows the user to move the view around and zoom in and out of the
tree visualization.

There are important challenges related to visualization. As programs
get larger, it will a good idea to be able to collapse parts of the
code. A strategy for collapsing a subtree is to attempt to evaluate it
and, if it evaluates to a value, then show this as a summary.  In
general, there will be certain challenges for visualization. If a
subtree can be collapsed, we want to ensure that the cursor is not in
that subtree, as the cursor would then be hidden. Similarly, if a
subtree is collapsed, the cursor should not be able to navigate into
that subtree.

A feature that will improve usability is to introduce a notion of
scrolling: make the default of the view to follow the cursor when
moving through the tree. There are a few different ways to do it. The
editor can follow the cursor with it being centered in the
view. Alternatively it can have the view follow whenever the cursor is
near the edge. For either of these cases, the editor could have a
button that moves the focus back to the cursor, making it possible to
toggle between the locked and the free view.

%%% Local Variables:
%%% mode: latex
%%% TeX-master: "../pepm2023"
%%% End:
