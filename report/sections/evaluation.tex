\section{Evaluation}
\label{evaluation}

This section will discuss how we handle evaluation of editor expressions and
ASTs. Editor expressions are evaluated in relation to an AST, and described by
the reduction semantics in the calculus \pepm. The reduction semantics is
divided into four different systems; an overall labeled transition system for
editor expressions, a big-step transition system for ASTs based on the editor
expression prefix \texttt{eval}, a labeled transition system for cursor
movement and substitutions based on the editor expression prefix $\pi$ where
$\pi \neq \texttt{eval}$, and finally a modal logic system for evaluating the
conditions in conditional editor expressions. Likewise for the implementation,
we have a hierarchy of four \elm{eval} functions as part of the code for
\elm{Edt}, \elm{Ast}, \elm{Aep} and \elm{Eed}, respectively. Whenever the user
decides to evaluate an editor expression upon their AST, the function
\elm{Edt.eval} is called, and from there all the other \elm{eval}s can be
invoked if necessary. The signatures of the said \elm{eval} functions are
\begin{lstlisting}[language=elm,%
                     label="eval-hiarchy",%
                     gobble=0,%
                     ]
{-Edt-}
eval : Completed -> WellFormedAst -> Evaluated
{-Aep-}
eval : Completed -> WellFormedAst -> AstOrVal
{-Eed-}
eval : Completed -> WellFormedAst -> Bool
{-Ast-}
eval : WellFormedAst -> Result EvalError Val
\end{lstlisting}
As such, we have a type-level guarantee from Elm that we can only apply an
editor expression if it is completed, and we can only apply an editor
expression to a well-formed AST. The rules, systems and the implementation will be
discussed in detail in the follow subsections.


\subsection{Evaluating editor expressions}
\begin{figure*}
  \center
  \renewcommand{\arraystretch}{2}
  \begin{tabular}{llll}
    \scriptsize(COND-1)  & $ \CondOne $           & \scriptsize(COND-2) & $ \CondTwo$ \\
    \scriptsize(EVAL)    & $ \Eval $              & \scriptsize(SEQ)    & $ \Seq$     \\
    \scriptsize(STRUCT)  & $\Struct$              & \scriptsize(CONTEXT)& \scriptsize$\Context$
  \end{tabular}
  \caption{Editor Expression reduction rules}
  \label{fig:edtreductionrules}
\end{figure*}

The system for reducing editor expressions is a labeled transition system
$(S_E, \mathcal{L}_E, \to)$, where $S_E = \edt \times \AST$ and $\mathcal{L}_E
= \val \cup \aep \cup \{\epsilon\}$ and where transitions take the form
$\breakpoint{E, a} \transition{$\alpha$} \breakpoint{E', a'}$ with $a$ being
well-formed, $E$ being completed and $\alpha \in \mathcal{L}_E$. \\

Editor expressions are always reduced in configurations of the form
$\breakpoint{E,a}$, where $E$ is the given editor expression and $a$ is the
currently build AST to which $E$ is applied. In terms of the labeled
transition system, we incrementally reduce the configuration until $E'$ becomes
$\nil$, and returns $\alpha$ for each increment. Figure
\ref{fig:edtreductionrules} show the different reduction rules for editor
expressions.

The rules \hyperref[fig:edtreductionrules]{(COND-1) and (COND-2)} describe that
for a biconditional editor expression, $\bicond{\phi}{E_1}{E_2}$, we either use
$E_1$ or $E_2$ as editor expression for the next reduction increment, based on
whether or not $\phi$ holds for the AST in the configuration. Hence it just
returns $\epsilon$. The rule \hyperref[fig:edtreductionrules]{(EVAL)} describes
that if we have the expression $\texttt{eval}.E$, we evaluate the AST to some
value in \val~which we return, and the expression of the new configuration is
$E$. The rule \hyperref[fig:edtreductionrules]{(SEQ)} describes that for a
composition $\seqcomp{E_1}{E_2}$ we reduce $E_1$ until it becomes $\nil$, while
updating the configuration and returning accordingly. The trivial case $E_1 =
\nil$ in sequential composition is caught by the rule
\hyperref[fig:edtreductionrules]{(STRUCT)}, as it describes that if two editor
expressions are equivalent, then two configurations using the editor expressions
respectively will also be reduced equivalently, e.g. we have that the sequential
composition $\seqcomp{\nil}{E_2}$ is equivalent to $E_2$ and thus reducing the
two expressing on an AST is also equivalent. This rule also takes care of
recursive editor expressions. The last rule
\hyperref[fig:edtreductionrules]{(CONTEXT)} describes that if we have a
substitution or cursor movement expression of the form $\pi.E$, then we reduce
the AST according to the labeled transition system for substitution and cursor
movement expressions, and update the configuration to contain the editor
expression $E$ and the new AST. \\

The implementation of $\elm{Edt.eval}$ has the same two arguments as the
reduction rules, i.e. a completed editor expression and a well-formed AST. The
return-value however is of the type $\elm{Evaluated}$, which is a type alias
over a record with two fields; a well-formed AST and a list of outputs after
reducing each editor expression.
\begin{lstlisting}[language=elm,%
                     label="eval-evaluated",%
                     gobble=0,%
                     ]
type alias Evaluated =
    { ast : WellFormedAst
    , output : List (Result Ast.EvalError Ast.Val)
    }
\end{lstlisting}
The $\elm{ast}$ field is used as the AST of the next configuration. That is,
when a editor expression is evaluated in a configuration, then a new
configuration will automatically be setup, where the AST is the one returned by
the last evaluation and the editor expression is uncompleted and constructed as
a $\elm{EdtHole}$. The $\elm{output}$ field is a list of $\elm{Result}$s such
that if we get an evaluation error, we return that error of type
$\elm{EvalError}$, and if not, we return the evaluation value of type
$\elm{Val}$\elmcore. Notice that we return a $\val$ if there is the editor
expression $\pi = \texttt{eval}$ and $\{\epsilon\}$ if not, thus
$\mathcal{L}_E\setminus\aep$.

In the implementation we need one more thing not accounted for by the
theoretical calculus; how we handle recursive bindings. We chose to use
explicit substituations of recursion variables and carry the extra information
of bindings around. For that we use a dictionary where the expression is bound
to its ID, expressed as the type $\elm{Dict String Completed}$. Since we do not
want to modify the signature of the calculus, $\elm{eval}$ instantiates an
empty $\elm{Dict}$ and calls $\elm{eval\_}$ to handle the reduction rules.
\begin{lstlisting}[language=elm,%
                   label="edt-eval",%
                   gobble=0,%
                   ]
{-Edt-}
eval e a =
    eval_ Dict.empty e a
\end{lstlisting}
$\elm{eval\_}$ unwraps the editor expression in a pattern match and applies the
rules accordingly. First are the two rules for prefix commands,
\hyperref[fig:edtreductionrules]{(EVAL) and (CONTEXT)}. Here we use the
function $\elm{Aep.eval}$ to evaluate the prefix itself, which we will get back
to after $\elm{Edt.eval}$. If it is a substitution or cursor movement it will
update the configuration and call $\elm{eval\_}$ recursively and if $\pi =
\texttt{eval}$ then it will concatenate the found value to the output of the
next expression.
\begin{lstlisting}[language=elm,%
                   label="edt-eval_-pre",%
                   gobble=0,%
                   ]
eval_ : Dict String Completed -> Completed -> WellFormedAst -> Evaluated
eval_ bindings e a =
    case e of
        Pre pi e1 ->
            case Aep.eval pi a of
                Aep.Ast a1 ->
                    eval_ bindings e1 a1

                Aep.Val v ->
                    (\{ ast, output } -> Evaluated ast (v :: output)) <|
                        eval_ bindings e1 a
\end{lstlisting}
Rules \hyperref[fig:edtreductionrules]{(COND-1) and (COND-2)} are straight forward.
\begin{lstlisting}[language=elm,%
                   label="edt-eval_-bicond",%
                   gobble=4,%
                   ]
    Bicond phi e1 e2 ->
        if Eed.eval phi a then
            eval_ bindings e1 a

        else
            eval_ bindings e2 a
\end{lstlisting}
Sequential composition can be written a bit more elegantly in code. Rather than
modeling the rules \hyperref[fig:edtreductionrules]{(SEQ) and (STRUCT)}
explicitly, we can do it implicitly by first reducing the first expression
and then the second expression on the AST obtained from reducing the first
expression. After that we return the AST of the second expression and the
outputs concatenated.
\begin{lstlisting}[language=elm,%
                   label="edt-eval_-seqcomp",%
                   gobble=4,%
                   ]
    SeqComp e1 e2 ->
        let
            r1 =
                eval_ bindings e1 a

            r2 =
                eval_ bindings e2 r1.ast
        in
            Evaluated r2.ast (r1.output ++ r2.output)
\end{lstlisting}
The same goes for recursive expressions, where instead of modeling
\hyperref[fig:edtreductionrules]{(STRUCT)} explicitly, we use the dictionary to
either bind or get the expression. We chose that calls to free variables
result in calls to the $\elm{Nil}$ expression.
\begin{lstlisting}[language=elm,%
                   label="edt-eval_-rec",%
                   gobble=8,%
                   ]
        Rec x e1 ->
            eval_ (Dict.insert (Var.toString x) e1 bindings) e1 a

        Call x ->
            (\e1 -> eval_ bindings e1 a)
                (Maybe.withDefault Nil <| Dict.get (Var.toString x) bindings)
\end{lstlisting}
For the expression $\elm{Nil}$ we do nothing and for the expression
$\elm{EdtHole}$ we use $\elm{never}$ as usual.
\begin{lstlisting}[language=elm,%
                   label="edt-eval_-nil",%
                   gobble=8,%
                   ]
        Nil ->
            Evaluated a []

        EdtHole n ->
            never n
\end{lstlisting}

The prefix command case use $\elm{Aep.eval}$. The return-type of the function is
$\elm{AstOrVal}$, which is a custom type that can be constructed by either a
well-formed AST or a  result-type value.
\begin{lstlisting}[language=elm,%
                   label="astorval",%
                   gobble=0,%
                   ]
type AstOrVal
    = Ast Ast.WellFormedAst
    | Val (Result Ast.EvalError Ast.Val)
\end{lstlisting}
The function itself simply unwraps the prefix command in a pattern match and
call either $\elm{Ast.substitute}$, $\elm{Ast.parent}$, $\elm{Ast.child}$ or
$\elm{Ast.eval}$ according to the case. All of these functions will be covered
in the following subsections, as they adhere to different reduction systems.
\begin{lstlisting}[language=elm,%
                   label="aep-eval",%
                   gobble=0,%
                   ]
{-Aep-}
eval pi a =
    case pi of
        Sub d ->
            Ast.substitute (Aam.getEquivalent d) a
                |> Ast

        Parent ->
            Ast.parent a
                |> Ast

        Child n ->
            Ast.child n a
                |> Ast

        Eval ->
            Ast.eval a |> Val

        AepHole n ->
            never n
\end{lstlisting}


\subsection{Evaluating abstract syntax trees}

\begin{figure*}[]
  \center
  \begin{tabular}{llll}
    \scriptsize(B-CONST)  & $ \BConst $         & \scriptsize(B-ABS)           & $ \BAbs$     \\
    \scriptsize(B-CURSOR) & $ \BCursor $        & \scriptsize(B-BREAKB)        & $ \BBreakB$  \\
    \scriptsize(B-APP)    & \scriptsize $\BApp$ &                              &              \\
    \scriptsize(B-APPB-1) & $ \BAppBOne $       & \scriptsize\text{(B-APPB-2)} & $ \BAppBTwo$ \\
    \scriptsize(B-APPH-1) & $ \BAppHOne $       & \scriptsize\text{(B-APPH-2)} & $ \BAppHTwo$
  \end{tabular}
  \caption{Big step transition rules for ASTs}
  \label{fig:transitionrulesast}
\end{figure*}

The system for evaluating ASTs is based on the big step semantics described as
(\AST, $\to$, \val) in Godiksen et al. \pepm, where transitions are of the form
$a \to v$ and express that the AST $a \in \AST$ evaluates to the value $v \in
\val$. The semantics change based on where $v$ is found, i.e. if $v \in \hval$
or $v \in \bval$. \\

The big-step semantics are listed in figure \ref{fig:transitionrulesast}. The
semantics works on ASTs and not cursor contexts and as such our \elm{eval}
function removes the cursor from its AST and passes it to another function
called \elm{evalAst}. Thus the rule
\hyperref[fig:transitionrulesast]{(B-CURSOR)} becomes obsolete and is omitted
from our implementation.

The function \elm{evalAst} handles the logic of the evaluation rules by pattern
matching on the root of the AST. In all cases except applications, the rules
state that the AST is already in normal form and hence simply returns its
corresponding $\val$. E.g. if the root of the AST is an unapplied
lambda-expression $\lambda x . a$, then applying the editor-expression
\elm{eval} returns $\val$ $\lambda x . a$.

Applications however, are not in normal form and have five different rules
describing the logic of reducing them. The rules depend on whether the value of
the branches of the AST application are in $\hval$ or $\bval$. The reduction is
described such that we evaluate the application function before the argument,
and the set where the result belongs dictates how or if we evaluate the
argument. If the function is in $\bval$ then we do not evaluate the argument, as
defined by rule \hyperref[fig:transitionrulesast]{(B-APPB-1)}. We handle this
order by first passing the AST application to a function named \elm{evalApp}.

This function composes the two functions over the monadic type \elm{Result}. The
\elm{Result} type encapsulates any errors that occur when encountering undefined
AST reduction behavior.

The function evaluates the first component of the application with
\elm{evalAst}, and then applies the result to the second component using
\elm{evalApp1}.

\elm{evalApp1} may call \elm{evalApp2} if we need to evaluate the second
parameter as well.

Lastly if we have the case \hyperref[fig:transitionrulesast]{(B-APP)} then
\elm{evalApp2} calls the function \elm{betaReduce} when handling the beta
reduction.

Thus \elm{eval} is implemented in a hierarchy of functions each consisting of a
single pattern match corresponding to matched clauses, where the functions have
the following signatures and where \elm{EvalError} is an error type representing
either applying a variable or constant.

\begin{lstlisting}[language=elm,%
                   label="eval-hierarchy",%
                   gobble=4,%
                   ]
    eval : WellFormedAst -> Result EvalError Val
    evalAst : Ast -> Result EvalError Val
    evalApp : Ast -> Ast -> Result EvalError Val
    evalApp1 : Val -> Ast -> Result EvalError Val
    evalApp2 : Var.Id -> Ast -> Val -> Val -> Result EvalError Val
    betaReduce : Var.Id -> Ast -> Ast -> Ast
\end{lstlisting}


\subsection{Substitution and cursor movement}

\begin{figure*}
  \center
  \renewcommand{\arraystretch}{2}
  \begin{tabular}{llll}
    \scriptsize(VAR)     & $ \Var $    & \scriptsize(CONST)     & $ \Const$   \\
    \scriptsize(HOLE)    & $ \Hole $   & \scriptsize(APP)       & $ \App$     \\
    \scriptsize(BREAK-1) & $\BreakOne$ & \scriptsize{(BREAK-2)} & $\BreakTwo$ \\
    \scriptsize(LAMBDA)  & $\LAMBDA $  &                        &
  \end{tabular}
  \caption{Substitution reduction rules}
  \label{fig:substitutionreductionrules}
\end{figure*}

The system for reducing substitution and cursor movement expressions is another
labeled transition system $(S_{\pi}, \mathcal{L}_{\pi}, \to)$, where $S_{\pi}
= \AST$ and $\mathcal{L}_E = \aep$ and where transitions take the form $a
\transition{$\pi$} a'$ with $a \in S_{\pi}$, $a' \in S_{\pi}$ and $\pi \in
\mathcal{L}_{\pi}$. \\

In the calculus \pepm, there are six possible node modifiers to choose from for
a substitution of transition $\pi$. Each of them are described by a single
reduction rule, except for substitution of breakpoints where there are two
reduction rules, \hyperref[fig:substitutionreductionrules]{(BREAK-1) and
(BREAK-2)}. The two breakpoint rules state the $\pi$ transition
$\{\texttt{break}\}$ respectively insert or remove a breakpoint between the
cursor and $\hat{a}$ in $a$. The other rules simply state that the subtree
$\hat{a}$ under the cursor in $a$ is substituted with the prefix transition
node modifier.

For the implementation we have written a function named \elm{substitute} (see
listing \ref{substitutefunction} in the appendix for the full version), that
takes an Ast and a well-formed AST as arguments. The former formal parameter
represents the atomic AST, and the latter the well-formed AST within which the
substitution takes place. We pattern match on the well-formed AST, rebuild the
cursor context and call the \elm{substitute} function recursively until we find
the cursor. The interesting part takes place when we find the cursor. That case
can be seen in the following listing.

\begin{lstlisting}[language=elm,%
                   label="substitute-at-cursor",%
                   gobble=4,%
                   ]
    Cursor a1 ->
        case ( a1, sub ) of
            ( Break a1_, Break _ ) ->
                Cursor a1_

            ( _, Break _ ) ->
                Cursor <| Break a1

            _ ->
                Cursor sub
\end{lstlisting}

Here we pattern match on the AST under the cursor in order to apply the correct
rule. The first case matches if
\hyperref[fig:substitutionreductionrules]{(BREAK-2)} holds, the second case
matches if \hyperref[fig:substitutionreductionrules]{(BREAK-1)} holds, and the
third case represents all other substitution rules. \\

\begin{figure*}
  \center
  \renewcommand{\arraystretch}{2}
  \begin{tabular}{llll}
    \scriptsize(APPC-1) & $\AppCOne$ & \scriptsize(APPC-2) & $\AppCTwo$ \\
    \scriptsize(APPP-1) & $\AppPOne$ & \scriptsize(APPP-2) & $\AppPTwo$ \\
    \scriptsize(ABSC)   & $\ABSC$    & \scriptsize{(ABSP)} & $\ABSP$    \\
    \scriptsize(BREAKC) & $\BreakC $ & \scriptsize{BREAKP} & $\BreakP$
  \end{tabular}
  \caption{Cursor movement reduction rules}
  \label{fig:cursorreductionrules}
\end{figure*}

Since cursor movement follows the same reduction system as substitution, both
the rules and the implementation are much akin to those of substitution. The
rules state that we find the cursor and then move it up or down the tree with
regards to the $\pi$ transition prefix.
\\

The implementation is divided into two functions; \elm{parent} that deals with
parent rules and \elm{child} that deals with child rules. Both functions
recursively find the cursor and rebuild a well-formed AST.

The \elm{parent} function (see listing \ref{parent-function} in the appendix for
the full version) pattern matches on the given well-formed AST. The interesting
cases are the listed below. The cases that we chose to exclude simply rebuild
the cursor context and call the function recursively.

\begin{lstlisting}[language=elm,%
                   label="parent-function",%
                   gobble=4,%
                   ]
    Cursor _ ->
        c

    CApp1 (Cursor a1) a2 ->
        Cursor <| App a1 a2

    CApp2 a1 (Cursor a2) ->
        Cursor <| App a1 a2

    CLambda x tau (Cursor a1) ->
        Cursor <| Lambda x tau a1

    CBreak (Cursor a1) ->
        Cursor <| Break a1
\end{lstlisting}

The first case matches if the cursor is already at the root. Without any type
system we cannot guarantee that someone calls this function with the cursor at
the root, so we just return the well-formed AST unchanged. The type system we
have implemented will however make sure this function is not called in this
case. The rest of the cases simply move the cursor from a child and up according
to their corresponding rule.

The \elm{child} function is divided into two by calling either \elm{childOne} or
\elm{childTwo}, depending on which child we want to navigate to.

\begin{lstlisting}[language=elm,%
                   label="child-function",%
                   gobble=4,%
                   ]
    child : Child -> WellFormedAst -> WellFormedAst
    child c =
        case c of
            One ->
                childOne

            Two ->
                childTwo
\end{lstlisting}

The \elm{childOne} and \elm{childTwo} functions (see listings
\ref{childOne-function} and \ref{childTwo-function} in the appendix for the full
versions) first find the cursor and rebuild the cursor context. After finding
the cursor, we pattern match on the AST under the cursor.

The pattern matching on the AST under the cursor in \elm{childOne} can be seen
in the following listing.

\begin{lstlisting}[language=elm,%
                   label="interesting-childOne-cases",%
                   gobble=4,%
                   ]
    case a of
        App a1 a2 ->
            CApp1 (Cursor a1) a2

        Lambda x tau a1 ->
            CLambda x tau <| Cursor a1

        Break a1 ->
            CBreak <| Cursor a1

        _ ->
            c
\end{lstlisting}

The first three cases handle rules \hyperref[fig:cursorreductionrules]{(APPC-1),
(ABSC) and (BREAKC)} respectively.

Similarly, the pattern matching on the AST under the cursor in \elm{childTwo}
can be seen in the following listing.

\begin{lstlisting}[language=elm,%
                   label="interesting-childTwo-cases",%
                   gobble=4,%
                   ]
    case a of
        App a1 a2 ->
            CApp2 a1 (Cursor a2)

        _ ->
            c
\end{lstlisting}


The first case handles rule \hyperref[fig:cursorreductionrules]{(APPC-2)}, where
we move the cursor to the second child of an application.

Neither \elm{childOne} nor \elm{childTwo} can guarantee that the functions are
not called when no valid child exists without a type system. As such, both
functions have a final case that matches anything, and return the well-formed
AST unchanged. The type system we have implemented will however guarantee that
these functions are not called in such cases.

\subsection{Conditions}

\begin{figure*}
  \center
  \renewcommand{\arraystretch}{2}
  \begin{tabular}{llllll}
    \scriptsize(AT-VAR)      & $\AtVar$      & \scriptsize(AT-CONST)  & $\AtConst$   & \scriptsize(AT-HOLE)   & $\AtHole$   \\
    \scriptsize(AT-APP)      & $\AtApp$      & \scriptsize(AT-ABS)    & $\AtAbs$     & \scriptsize(AT-BREAK)  & $\AtBreak$  \\
    \scriptsize(POS-TRIVIAL) & $\PosTrivial$ & \scriptsize(POS-ABS)   & $\PosAbs$    & \scriptsize(POS-BREAK) & $\PosBreak$ \\
    \scriptsize(POS-APP-1)   & $\PosAppOne$  & \scriptsize(POS-APP-2) & $\PosAppTwo$ &   \scriptsize(NEC-TRIVIAL) & $\NecTrivial$ \\
    \scriptsize(NEC-APP)     & $\NecApp$     & \scriptsize(NEC-ABS)   & $\NecAbs$    & \scriptsize(NEC-BREAK) & $\NecBreak$
  \end{tabular}
  \caption{Condition reduction rules}
  \label{fig:conditionreductionrules}
\end{figure*}

The system for reducing logical conditions is simply that of modal logic with
the addition of three modalities; $\at$ (pronounced "at"), $\possibly$
(pronounced "possibly" or "diamond") and $\necessarily$ (pronounced
"necessarily" or "box"). The rules are written in figure
\ref{fig:conditionreductionrules}. We use the modality $\at$ as the trivial
logical case, where an arbitrary AST node models $\at$ at the corresponding
node modifier, e.g. $\hole \models \at \text{hole}$. The semantics of the
trivial cases are described by the rules
\hyperref[fig:conditionreductionrules]{(AT-...)}. For the modality $\possibly$
we first check if any of the trivial $\at$ rules hold for the given AST node
and node modifier, described in rule
\hyperref[fig:conditionreductionrules]{(POS-TRIVIAL)}. If it is not the trivial
case, but the AST node is an abstraction, breakpoint or application, we check
if $\possibly$ holds for the subtrees beneath, described in rules
\hyperref[fig:conditionreductionrules]{(POS-ABS), (POS-BREAK), (POS-APP-1) and
(POS-APP-2)}. Notice here that for applications, $\possibly$ only needs to hold
for one of the AST node children in order to satisfy $\possibly \text{app}$. A
condition with the $\necessarily$ modality holds under the same clauses as
$\possibly$, since it is either trivial or holds for an AST node at which its
children satisfy $\possibly$, described in the four
\hyperref[fig:conditionreductionrules]{(NEC-...)} rules. However,
$\possibly\text{app}$ and $\necessarily\text{app}$ is handled differently in
that $\possibly$ holds when $\possibly$ holds for one of the AST node children,
whereas $\necessarily$ holds when $\possibly$ holds for both the AST node
children. \\

In the implementation, the $\elm{eval}$ function for reducing $\pi$ editor
expressions consists of a pattern match over the given condition $\phi$ of type
$\elm{Eed}$. The match will reduce negation, conjunction and disjunction by
modal logic semantics and recursive calls, and the match will reduce an
expression hole with the $\elm{never}$ function as usual. Matching to the
condition $@$, $\possibly$ or $\necessarily$ is handled by extracting the
subtree at the cursors location in the given AST and then calling either the
function $\elm{isEquivalent}$, $\elm{possibly}$ or $\elm{necessarily}$ on the
subtree and the node modifier of the condition.
\begin{lstlisting}[language=elm,%
                   label="eval-eed",%
                   gobble=0,%
                   ]
{-Eed-}
eval phi a =
    case phi of
        Neg phi1 ->
            not <| eval phi1 a

        Conjunction phi1 phi2 ->
            eval phi1 a && eval phi2 a

        Disjunction phi1 phi2 ->
            eval phi1 a || eval phi2 a

        At d ->
            Aam.isEquivalent <| Ast.underCursor a

        Possibly d ->
            possibly d <| Ast.underCursor a

        Necessarily d ->
            necessarily d <| Ast.underCursor a

        EedHole n ->
            never n
\end{lstlisting}
~

The three functions $\elm{isEquivalent}$, $\elm{possibly}$ and
$\elm{necessarily}$ all have the signature
\begin{lstlisting}[language=elm,%
                   label="eval-eed-signature",%
                   gobble=0,%
                   ]
Aam.Completed -> Ast -> Bool
\end{lstlisting}
The node modifier is completed, but the AST is not well-formed since it
is the AST under the cursor from the last function. Hence we only expose
$\elm{eval}$ and none of the other functions, such that we still can only
evaluate completed editor expressions and well-formed ASTs with a type-level
guarantee from Elm. \\

For the $\elm{isEquivalent}$ function the
\hyperref[fig:conditionreductionrules]{(AT-...)} rules are implemented by a
pattern match over the parameters.
The function applies the rules very straightforwardly such that if one of the
clauses holds then it returns $\elm{True}$ and if not then it returns
$\elm{False}$, where the only addition is the usual $\elm{never}$ function
for a node modifier hole. In rule
\hyperref[fig:conditionreductionrules]{(AT-CONST)} it is not clear whether the
constants have to be equal or just some constant in order for $\at(\const c)$ to
hold. We chose that the constants must be equal.
For all three of the functions, we discard unused parameters of the term
constructors, such as variable IDs and type annotations.
\begin{lstlisting}[language=elm,%
                   label="eval-aam-isequivalent",%
                   gobble=0,%
                   ]
isEquivalent d a =
    case ( d, a ) of
        ( Var _, Ast.Var _ ) ->
            True

        ( Const c1, Ast.Const c2 ) ->
            c1 == c2

        ( App _ _, Ast.App _ _ ) ->
            True

        ( Lambda _ _ _, Ast.Lambda _ _ _ ) ->
            True

        ( Break, Ast.Break _ ) ->
            True

        ( Hole _, Ast.Hole _ ) ->
            True

        ( AamHole n, _ ) ->
            never n

         _ ->
             False

\end{lstlisting}

The $\elm{possibly}$ function is reduced by first checking whether we have the
trivial case \hyperref[fig:conditionreductionrules]{(POS-TRIVIAL)}. If we do
not, we check whether any of the other rules holds, i.e. if we are at an
application, abstraction or breakpoint, it calls recursively on the children of
the AST node. It deviates from the rules in that it combine the rules
\hyperref[fig:conditionreductionrules]{(POS-APP-1) and (POS-APP-2)} by a logical
or-operator.
\begin{lstlisting}[language=elm,%
                   label="eval-eed-possibly",%
                   gobble=4,%
                   ]
    possibly d a =
        (case a of
            Ast.App a1 a2 ->
                possibly d a1 || possibly d a2

            Ast.Lambda _ _ a1 ->
                possibly d a1

            Ast.Break a1 ->
                possibly d a1

            _ ->
                False
        )
            || Aam.isEquivalent d a
\end{lstlisting}
~
The $\elm{necessarily}$ function is implemented in the same manner as
$\elm{possible}$, with the only difference being if it is not the trivial case
and the AST node is an application, then it uses a logical and-operator instead
of a logical or-operator.
\begin{lstlisting}[language=elm,%
                   label="eval-eed-necessarily",%
                   gobble=0,%
                   ]
necessarily d a =
    (case a of
        Ast.App a1 a2 ->
            possibly d a1 && possibly d a2
        Ast.Lambda _ _ a1 ->
            possibly d a1

        Ast.Break a1 ->
            possibly d a1

        _ ->
            False
    )
        || Aam.isEquivalent d a
\end{lstlisting}
