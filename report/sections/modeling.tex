\section{Modeling}
\label{modeling}

In this section we will discuss the design decisions we have taken when modeling
the editor. We have focused on leveraging the types of Elm to model the editor
calculus as closely as possible without introducing additional constructions.

\subsection{Set and Ctx}

Before we continue with the discussion on modeling, we will go on a short
tangent on two custom types we have defined, since we use these types throughout
this section and section \ref{type-checking}.

In Elm, we have a type constructor \elm{Set a} that represents mathematical
sets. This type is defined in a module along with functions for working with
this type. Most of these functions require the elements \elm{a} of a \elm{Set a}
to be an instance of the \elm{comparable} type class. The problem is that Elm
does not allow making types instances of type classes (instead there are a few
predefined type classes and instances), and as such, we are very limited in what
can put in such sets. Because of this we decided to implement our own \elm{Set
a} type constructor, as seen in listing \ref{listing:extra-set}, in the module
\elm{Extra.Set}. Whenever we refer to \elm{Set a} in this paper, we refer to our
custom type.

\begin{lstlisting}[language=elm,%
                   label={listing:extra-set},%
                   gobble=4,%
                   caption={Our own \elm{Set a}.},%
                   ]
    type Set a
        = Set (List a)
\end{lstlisting}

A similar issue arises with the \elm{Dict k v} type constructor. \elm{Dict k v}
is a data structure that stores associations between keys \elm{k} and values
\elm{v}. Most functions that work with \elm{Dict k v} require the type of the
key to be an instance of the \elm{comparable} type class. We have implemented
our own version of \elm{Dict k v}, as seen in listing \ref{listing:ctx}, but we
have called it \elm{Ctx a b}, as we use it for type contexts.

\begin{lstlisting}[language=elm,%
                   label={listing:ctx},%
                   gobble=4,%
                   caption={Our own version of \elm{Dict k v}.},%
                   ]
    type Ctx a b
        = Ctx (List ( a, b ))
\end{lstlisting}

\subsection{Abstract syntax trees}
Abstract syntax trees are defined by the formation rules (\ref{formationA}),
and as such, each node is either a variable, constant, application,
abstraction, cursor, breakpoint or hole.
\begin{align}
    \Ast\label{formationA}
\end{align}

We can model ASTs using custom types, where every formation rule
in equation (\ref{formationA}) from Godiksen et al. $\pepm$ has its term
constructor. However, we will omit the cursor term constructor, as we will
introduce a cursor through a cursor context in the next subsection.

\begin{lstlisting}[language=elm,%
                   label={ast-definition},%
                   gobble=4,%
                   caption={Formation rules (\ref{formationA}) modeled in Elm},%
                   ]
    type Ast
        = Var Var.Id
        | Const Const.Value
        | App Ast Ast
        | Lambda Var.Id Ast
        | Break Ast
        | Hole
\end{lstlisting}

Most of the term constructors take arguments that contain the information of
either the given node or its children, depending on if it is a leaf or not. From
these constructor arguments, we introduce two new types; variable IDs and
constant values. These are not defined in the calculus of which this project is
based on \pepm, and thus they are implementation specific.

Variables are implemented as a custom type named \elm{Id}, where its constructor takes a \elm{String}~as argument.

\begin{lstlisting}[language=elm,%
                   gobble=0,%
                   ]
type Id = Id String
\end{lstlisting}

Constant values are more interesting in that they dictate what constant types
the editor should work with. We have created a \elm{Const}-module with
extensibility in mind such that it can easily be extended to support more
constant types by adding new term constructors and pattern matching cases.
However, we have only implemented support for integers for this project.

\begin{lstlisting}[language=elm,%
                   gobble=0,%
                   ]
type Value = IntVal Int
\end{lstlisting}

\subsection{Cursor contexts}

A cursor context C is given by the formation rules in equation
(\ref{formationC1}) and (\ref{formationHat}), where $\hat{a}$ is an AST without
a cursor. A well-formed AST is formed by $C[\dot{a}]$, where well-formedness is
defined by the AST having exactly one cursor. The formation rule for $\dot{a}$
is that of equation (\ref{formationDot}).
\begin{align}
    \C \label{formationC1}    \\
    \aHat\label{formationHat} \\
    \aDot\label{formationDot}
\end{align}

In Godiksen et al. \pepm, the cursor context is defined in order to both locate
a cursor within an AST, and to ensure that one, and only one cursor exists
within an AST.

When implementing the editor, we redesigned the formation for $\dot{a}$ such
that it only has the clause $\llbracket\hat{a}\rrbracket$. This is equivalent to
the cursor context defined in the paper since all other clauses of $\dot{a}$ can
be formed by a combination of $C$ and $\llbracket\hat{a}\rrbracket$. Furthermore
we model $\dot{a}$ implicitly in the formation of $C$ by substituting the first
clause of $C$ by the only clause of the new $\dot{a}$. As such, we have modeled
the cursor context using custom types after the formation rules of equation
(\ref{formationC2}) and (\ref{formationHat}). This is simpler in implementation
since we only have two data types instead of three.

\begin{align}\label{formationC2}
    C ::= \llbracket \hat{a} \rrbracket
    \mid \app{C}{\hat{a}}
    \mid \app{\hat{a}}{C}
    \mid \abs{x}{C}
    \mid \breakpoint{C}
\end{align}

Originally we decided against modeling the cursor context directly and instead
did it indirectly with the way we defined ASTs. Firstly, we added the cursor as
a term constructor for ASTs.

\begin{lstlisting}[language=elm,%
    gobble=0,%
    ]
type Ast
    = Var Var.Id
    | Con Const.Value
    | App Ast Ast
    | Lambda Var.Id Ast
    | Break Ast
    | Hole
    | Cursor Ast
\end{lstlisting}

We then created the opaque type \elm{WellFormedAst} which encapsulated an
AST. Since the type was opaque, we had full control over it in the \elm{Ast}
module, and could guarantee that an AST that was not well-formed would never be
injected into the \elm{WellFormedAst} type.

\begin{lstlisting}[language=elm,%
    gobble=0,%
    ]
type WellFormedAst
    = WellFormed Ast
\end{lstlisting}

We then created a function that returned an ``empty'' \elm{WellFormedAst},
i.e. a cursor with a hole inside. This assured us that an AST always started out
as well-formed.

We could then have any exposed function that worked with ASTs take a
\elm{WellFormedAst} and know that it would remain well-formed as we know from
the editor calculus by Godiksen et al. \pepm.

In order to locate the cursor in an AST, we would simply traverse the AST to
find the cursor. Because the AST was well-formed, we knew that we would find a
cursor, and only one. After the cursor was located, we could do any operation at
that location, such as substituting what was under the cursor.

% Explain why this is a bad approach

We eventually decided against this approach, because it made it more complicated
to find the cursor, as we potentially have to traverse through the entire AST to
find it. Secondly, it generally just made more sense to keep the guarantees from
the calculus by modeling it directly. \\

% Explain the direct modeling approach

To model the cursor context directly, we model the cursor context type as seen
in listing \ref{ast-definition} and \ref{cursor-context-formation-elm}.

\begin{lstlisting}[language=elm,%
                   label={cursor-context-formation-elm},%
                   gobble=4,%
                   caption={Formation rules (\ref{formationC2}) modeled in Elm},%
                   ]
    type C
        = Cursor Ast
        | CApp1 C Ast
        | CApp2 Ast C
        | CLambda Var.Id C
        | CBreak C
\end{lstlisting}


To ensure that an AST is well-formed, we have defined a module for ASTs where
its term constructors for $\hat{a}$ are exposed but the cursor context $C$ is an
opaque type. This means that an AST can always be constructed outside the
module, but a cursor context cannot be constructed outside the module such that
we have full control over how it is constructed and manipulated. We then expose
some functions that construct well-formed ASTs and require that all functions
with ASTs as parameters must take a well-formed AST. These functions are
implemented using the reduction rules from the paper, and as such (given that
the functions are correctly implemented) return well-formed ASTs.
\begin{lstlisting}[language=elm,%
                   label={cursor-context-alias-elm},%
                   gobble=0,%
                   caption={Type alias for well-formed ASTs in Elm},%
                   ]
type alias WellFormedAst = C
\end{lstlisting}


For accessing the AST under the cursor, we define the function
\texttt{underCursor}. The function takes a well-formed AST and returns the AST
under the cursor.

\begin{lstlisting}[language=elm,%
    label="underCursor",%
    gobble=4,%
    ]
    underCursor : WellFormedAst -> Ast
    underCursor c =
        case c of
            Cursor a ->
                a

            CApp1 c1 _ ->
                underCursor c1

            CApp2 _ c2 ->
                underCursor c2

            CLambda _ c1 ->
                underCursor c1

            CBreak c1 ->
                underCursor c1
\end{lstlisting}

\subsection{Editor expressions}

Editor expressions are directly modeled using the formation rules given in the
article by Godiksen et al.\pepm.

\begin{align}
    \Aep \label{aep-formation-rules} \\
    \Eed \label{eed-formation-rules} \\
    \Edt \label{edt-formation-rules} \\
    \Aam \label{aam-formation-rules}
\end{align}

As an example, in equation (\ref{aep-formation-rules}) we have four rules for $\pi$; one
for evaluation, one for substitution and two for cursor navigation. We model
this as a custom type with four term constructors:

\begin{lstlisting}[language=elm,%
                   label={aep-without-holes-definition},%
                   gobble=4,%
                   caption={Formation rules (\ref{aep-formation-rules}) modeled in Elm},%
                   ]
    type Aep
        = Eval
        | Sub Aam
        | Child Ast.Child
        | Parent
\end{lstlisting}

The rest of the formation rules are defined similarly, as seen in figures
\ref{eed-without-holes-definitions}, \ref{edt-without-holes-definitions} and
\ref{aam-without-holes-definitions}.

\begin{lstlisting}[language=elm,%
           label={eed-without-holes-definitions},%
           gobble=4,%
           caption={Formation rules (\ref{eed-formation-rules}) modeled in Elm},%
           ]
    type Eed
        = Neg Eed
        | Conjunction Eed Eed
        | Disjunction Eed Eed
        | At Aam
        | Possibly Aam
        | Necessarily Aam
\end{lstlisting}

\begin{lstlisting}[language=elm,%
           label={edt-without-holes-definitions},%
           gobble=4,%
           caption={Formation rules (\ref{edt-formation-rules}) modeled in Elm},%
           ]
    type Edt
        = Pre Aep Edt
        | Bicond Eed Edt Edt
        | SeqComp Edt Edt
        | Rec Var.Id Edt
        | Call Var.Id
        | Nil
\end{lstlisting}

\begin{lstlisting}[language=elm,%
           label={aam-without-holes-definitions},%
           gobble=4,%
           caption={Formation rules (\ref{aam-formation-rules}) modeled in Elm},%
           ]
    type Aam
        = Var Var.Id
        | Con Const.Value
        | App
        | Lambda Var.Id
        | Break
        | Hole
\end{lstlisting}

Some term constructors take arguments. A \elm{Sub} constructor, for example,
takes an \elm{Aam}, and a \elm{Child} constructor takes an AST child. Of these
arguments, we have just one new type that is not yet modeled; AST children.

AST children have the type $T \in \ctyp$, and are formed by the formation rules
(\ref{formationchild}). A child is defined in relation to some arbitrary AST
node, and can either be the first or the second child of that node. In this case, if we
have the editor expression $\pi = \texttt{child } n$, then $n \in \ctyp$ is
defined in relation to the cursor. AST children are also used again later for the type system.

\begin{equation}\label{formationchild}
    \Ctyp
\end{equation}

We have modeled it in Elm with the custom type

\begin{lstlisting}[language=elm,%
           label={child-formation-elm},%
           gobble=0,%
           caption={Formation rules (\ref{formationchild}) modeled in Elm},%
           ]
type Child
    = One
    | Two
\end{lstlisting}

In order to evaluate editor expressions, specifically the expression $\pi =
    \texttt{eval}$, we have to introduce and model three more types (the evaluation
itself is discussed in section \ref{evaluation}). The return value of evaluating
this expression is of type \val. The value, $v \in \val$, may also contain
$\hval$ or $\bval$. $\hval$ and $\bval$ represents evaluated ASTs that contains
holes and evaluated ASTs that contains breakpoints, respectively, while only
being found in \val~represents all other evaluated ASTs. The formation rules for
the values, where $v \in \val$, $u \in \hval$ and $w \in \bval$, are

\begin{align}
    \Val \label{val-formation-rules}   \\
    \HVal \label{hval-formation-rules} \\
    \BVal \label{bval-formation-rules}
\end{align}

The formation rules are modeled in Elm using custom types --- as seen in
figures \ref{Val-formation}, \ref{HVal-formation} and \ref{BVal-formation} ---
in a similar style to how other formation rules have been modeled.

\begin{lstlisting}[language=elm,%
                     label={Val-formation},%
                     gobble=6,%
                     caption={Formation rules (\ref{val-formation-rules}) modeled in Elm},%
                     ]
      type Val
        = VConst Const.Value
        | VLambda Var.Id Ast
        | W BVal
        | U HVal
\end{lstlisting}

\begin{lstlisting}[language=elm,%
                     label={HVal-formation},%
                     gobble=6,%
                     caption={Formation rules (\ref{hval-formation-rules}) modeled in Elm},%
                     ]
      type HVal
        = HHole
        | HApp1 HVal Val
        | HApp2 Var.Id Ast HVal
\end{lstlisting}

\begin{lstlisting}[language=elm,%
                     label={BVal-formation},%
                     gobble=6,%
                     caption={Formation rules (\ref{bval-formation-rules}) modeled in Elm},%
                     ]
      type BVal
        = BBreak Ast
        | BApp1 BVal Ast
        | BApp2 Var.Id Ast BVal
\end{lstlisting}

\subsection{Types}

We use the variable $\tau$ to denote the type of an AST, where $\tau \in \atyp$.
The type can be that of a constant (called the base type), that of an
abstraction, that of a breakpoint or some not yet determined type (called the
indeterminate type). The formation rules are

\begin{equation}\label{eq:atyp-formation}
    \Atyp
\end{equation}

We have modeled the formation rules in Elm using a custom type, where
the term constructors correspond to the formation clauses.

\begin{lstlisting}[language=elm,%
                   gobble=0,%
                   caption={Formation rules (\ref{eq:atyp-formation}) modeled in Elm},%
                   ]
type ATyp
    = TBase
    | TLambda ATyp ATyp
    | TBreak ATyp
    | TIndeterminate
\end{lstlisting}

With regards to the type checker, Godiksen et al. make some alterations to the
formation rules of AST node modifiers as well as AST nodes.\pepm~For node
modifiers we include type annotations for applications, abstractions and holes,
while for ASTs we include type annotations for abstraction parameters and
holes. The alterations are defined in equation (\ref{eq:lan-mod-aam}) and
(\ref{eq:lan-mod-ast}).
\begin{equation}\label{eq:lan-mod-aam}
  D ::= \var{x}
  \mid \const{c}
  \mid \bind{\aamApp}{\tau_1\to\tau_2,\tau_1}~~~~~
\end{equation}
\begin{equation*}~~~~
  \mid \bind{\aamLambda{x}}{\tau_1\to\tau_2}
  \mid \aamBreak
  \mid \bind{\aamHole}{\tau}
\end{equation*}
\begin{equation}\label{eq:lan-mod-ast}
  a ::= x
  \mid c
  \mid \app{a_1}{a_2}
  \mid \abs{\bind{x}{\tau}}{a}
  \mid \cursor{a}
  \mid \breakpoint{a}
  \mid \bind{\hole}{\tau}
\end{equation}

For the implementation we extend the custom type with $\elm{ATyp}$ arguments to
some term constructors to reflect the new formation rules. We also shorten the
type annotation from $D = \bind{\aamApp}{\tau_1\to\tau_2,\tau_1}$ to
$\bind{\aamApp}{\tau_1\to\tau_2}$. The reason for this being that we can extract
$\tau_1$ using a pattern match in the implementation, hence it is redundant
information.

\begin{lstlisting}[language=elm,%
    gobble=0,%
    caption={Formation rules (\ref{eq:lan-mod-aam}) modeled in Elm}%
    ]
type Aam
    = Var Var.Id
    | Const Const.Value
    | App ATyp ATyp
    | Lambda Var.Id ATyp ATyp
    | Break
    | Hole ATyp
\end{lstlisting}
\begin{lstlisting}[language=elm,%
    gobble=0,%
    caption={Formation rules (\ref{eq:lan-mod-ast}) modeled in Elm}%
    ]
type Ast
    = Var Var.Id
    | Const Const.Value
    | App Ast Ast
    | Lambda Var.Id ATyp Ast
    | Break Ast
    | Hole ATyp
type C
    = Cursor Ast
    | CApp1 C Ast
    | CApp2 Ast C
    | CLambda Var.Id ATyp C
    | CBreak C
\end{lstlisting}

We also introduce paths and two kinds of type contexts used for type checking.
Paths, denoted as $p \in \pth$, use the already defined child types $T \in
    \ctyp$ to specify a path from the root-node of some tree to some node in its
subtree. Paths are formed by the two terms

\begin{equation}\label{eq:path-formation}
    \Pth
\end{equation}

We have modeled paths as a type constructor, such that a path does not depend on
$\ctyp$, but instead is a generic path that takes a definition of children as an
argument. This allows paths to be extended beyond only denoting AST paths, which
may be useful in the future.

\begin{lstlisting}[language=elm,%
                   label={pth-type},%
                   gobble=0,%
                   caption={Formation rules (\ref{eq:path-formation}) modeled in Elm},%
                   ]
type Pth t
    = Child (Pth t) t
    | Epsilon
\end{lstlisting}

The two type contexts are AST type contexts, denoted $\ctx{a} \in \actx$, and
editor expression type contexts, denoted $\ctx{e} \in \ectx$. An AST type
context stores type bindings for variables in an AST.

\begin{equation}\label{eq:actx}
    \actx = \textbf{Var} \rightharpoonup \atyp
\end{equation}

An editor expression type context stores for all paths with a known subtree in
an AST, a pair of an AST type context and an AST type.

\begin{equation}\label{eq:ectx}
    \ectx = \pth \rightharpoonup (\actx \times \atyp)
\end{equation}

We implemented the two contexts using type aliases to our \elm{Ctx a b} type. An
AST type context has variables as keys and \elm{ATyp} as values.

\begin{lstlisting}[language=elm,%
                   gobble=0,%
                   caption={Equation (\ref{eq:actx}) modeled in Elm},%
                   ]
type alias ACtx =
    Ctx Var.Id ATyp
\end{lstlisting}

Editor expression type contexts have paths as keys and tuples of the form
\elm{( ACtx, ATyp )} as values.

\begin{lstlisting}[language=elm,%
                   gobble=0,%
                   caption={Equation (\ref{eq:ectx}) modeled in Elm},%
                   ]
type alias ECtx =
    Ctx Ast.Pth ( Ast.ACtx, Ast.ATyp )
\end{lstlisting}

\subsection{Editor expression builder}

To build editor expressions, we need to introduce holes to the editor
expressions. We define a hole term constructor for each type of editor
expression, as seen in figure \ref{fig:editorexpressionswithholes}

\begin{figure}[H]
    \center
    \begin{tabular}{llll}

        \begin{lstlisting}[language=elm,%
                            gobble=8,%
                            mathescape,%
                            ]
             type Aep
                = Eval
                    $\vdots$
                | AepHole
        \end{lstlisting} &

        \begin{lstlisting}[language=elm,%
                            gobble=8,%
                            mathescape,%
                            ]
            type Eed
                = Neg Eed
                     $\vdots$
                | EedHole
        \end{lstlisting} \\

        \begin{lstlisting}[language=elm,%
                            gobble=8,%
                            mathescape,%
                            ]
            type Edt
                = Pre Aep Edt
                    $\vdots$
                | EdtHole
        \end{lstlisting} &

        \begin{lstlisting}[language=elm,%
                            gobble=8,%
                            mathescape,%
                            ]
            type Aam
                = Var Var.Id
                    $\vdots$
                | AamHole
        \end{lstlisting}
    \end{tabular}
    \caption{Editor expression definitions with holes}
    \label{fig:editorexpressionswithholes}
\end{figure}

We are now able to build editor expressions by initializing the editor
expression builder with an \texttt{EdtHole}, and then allow the user to
substitute holes with appropriate expressions. As with ASTs, we have the concept
of atomic editor expressions, which are editor expressions with holes as
children. The user can substitute holes until no holes are left. An editor
expression without any holes is said to be \textit{completed}.

We are only interested in evaluating completed editor expressions. Therefore, we
need to know when an editor expression is completed, and only then allow the
user to evaluate it.

We considered two approaches. The first approach would be to introduce an
opaque type with a single term constructor. For example

\begin{lstlisting}[language=elm,%
                   label="completed-edt-definition",%
                   gobble=4,%
                   ]
    type CompletedEdt = Completed Edt
\end{lstlisting}

We would also need a function that takes an \texttt{Edt}, recursively
checks whether it contains any holes, and returns a \texttt{Maybe CompletedEdt}.
We will refer to this function as \texttt{toCompleted} in this subsection.

With the \texttt{CompletedEdt} type and the \texttt{toCompleted} function we can
create some guarantees. For example, if the editor expressions evaluation
function takes a \texttt{CompletedEdt} instead of an \texttt{Edt}, we ensure ---
given that \texttt{toCompleted} is properly implemented --- that uncompleted
editor expressions cannot be evaluated.

This approach would work, but has some drawbacks. Since we introduce the
\texttt{CompletedEdt} type, we need to convert between \texttt{Edt} and
\texttt{CompletedEdt} a lot, which can clutter the code. More critically, this
approach depends on the constructor of \texttt{CompletedEdt} not being exposed
and \texttt{toCompleted} to be correctly implemented. This is not a serious
issue, but we can do better.

We decided to move away from this first approach and rather introduce a type
variable to each editor expression type. The type variable would be used
for recursively defining the type in terms of the type variable, but also as an
argument to each hole constructor, as seen in the following listings.

\begin{lstlisting}[language=elm,%
                   label="aep-definition",%
                   gobble=4,%
                   ]
    type Aep a
        = Eval
        | Sub (Aam a)
        | Child Ast.Child
        | Parent
        | AepHole a
\end{lstlisting}

\begin{lstlisting}[language=elm,%
                   label="eed-definitions",%
                   gobble=4,%
                   ]
    type Eed a
        = Neg (Eed a)
        | Conjunction (Eed a) (Eed a)
        | Disjunction (Eed a) (Eed a)
        | At (Aam a)
        | Possibly (Aam a)
        | Necessarily (Aam a)
        | EedHole a
\end{lstlisting}

\begin{lstlisting}[language=elm,%
                   label="generic-edt-definition",%
                   gobble=4,%
                   ]
    type Edt a
        = Pre (Aep a) (Edt a)
        | Bicond (Eed a) (Edt a) (Edt a)
        | SeqComp (Edt a) (Edt a)
        | Rec Var.Id (Edt a)
        | Call Var.Id
        | Nil
        | EdtHole a
\end{lstlisting}

\begin{lstlisting}[language=elm,%
                   label="aam-definitions",%
                   gobble=4,%
                   ]
    type Aam a
        = Var Var.Id
        | Con Const.Value
        | App ATyp ATyp
        | Lambda Var.Id ATyp ATyp
        | Break
        | Hole ATyp
        | AamHole a
\end{lstlisting}

We can utilize the \texttt{()} and the \texttt{Never} types to represent
completed and uncompleted editor expressions. If for example \texttt{a} in
\texttt{Edt a} is replaced with \texttt{()}, the editor expression can contain
holes. Conversely, if we replace \texttt{a} with \texttt{Never}, it cannot
contain holes, since we need a \texttt{Never} value to construct a hole. We can
therefore use \texttt{Edt ()} to represent uncompleted editor expressions, and
\texttt{Edt Never} to represent completed editor expressions. We have created a
type alias for completed and uncompleted editor expressions of every type. For
example, the following are the type aliases for \texttt{Edt a}.

\begin{lstlisting}[language=elm,%
                   label="completed-and-uncompleted-edts",%
                   gobble=4,%
                   ]
    type alias Uncompleted =
        Edt ()

    type alias Completed =
        Edt Never
\end{lstlisting}

This approach creates stronger guarantees by making impossible states
impossible. The \texttt{toCompleted} function cannot take an
\texttt{uncompleted} editor expression, return a \texttt{completed} editor
expression and still have bugs, since we have constrained the types to create
type level guarantees.


\subsection{Editor expression builder substitutions}

We need a way for the user to build an expression from a hole to a completed
editor expression. In order to do this, we want to show any holes that might be
in the editor expression on the user interface, and allow the user to choose a
substitution.

We have some modeling design decisions to consider for this to work. As we are working
with Elm, we need to implement a \texttt{view} function that returns the HTML
for rendering the editor expression we are building. When we return the
HTML for a hole, we want it to be a button element that, when pressed, sends a
message to an \texttt{update} function. This message needs to somehow specify
which hole is to be substituted. There are many approaches we could have taken
and we will now discuss two approaches we have considered, and why we chose
the approach we did.

There are four types of holes that we can potentially substitute, i.e. one for
an \texttt{Edt}, \texttt{Aep}, \texttt{Aam} and an \texttt{Eed}. As such, we
need one message constructor for each type of hole, since they all require their
own type of substitution expression. We have defined the following message type
and constructors for our substitutions which we will extend later.

\begin{lstlisting}[language=elm,%
                   gobble=4,%
                   ]
    type Msg
        = SubstituteEdt
        | SubstituteAep
        | SubstituteAam
        | SubstituteEed
\end{lstlisting}

One approach would be to build up a path as we recursively build the HTML for
the editor expression, and then have the message contain that path. I.e. if we
use the path type constructor from listing \ref{pth-type}, and give it a
definition for an editor expression child type, we could extend our message type
with this path giving us the following type.

\begin{lstlisting}[language=elm,%
                   gobble=4,%
                   ]
    type Msg
        = SubstituteEdt (Pth Child)
        | SubstituteAep (Pth Child)
        | SubstituteAam (Pth Child)
        | SubstituteEed (Pth Child)
\end{lstlisting}

When the \texttt{update} function receives such a message, it knows which type
of substitution is required, and it knows the path to where the substitution is
to be made.

It would then simply traverse the editor expression using the path, and
substitute what the path points to with a valid substitution. However, from the
perspective of the \texttt{update} function, it has no idea of whether the path
is valid. What the \texttt{update} function should do if the path does not exist
in the editor expression is unclear, and as such we would have to handle that
error case. Furthermore, if the path points to an element that does not match
the expected substitution type we would also have to handle that error. Since we
control both the \texttt{view} and the \texttt{update} functions, we can be
fairly certain that such errors would never happen, but it would be better if we
could make that a guarantee through the types of Elm.

This leads us to the second approach we have considered. It is a bit more
involved, but does not contain the flaws of the previous approach. The idea is
to build up a function as we build up the HTML for the editor expression. This
function is a function that when given a substitution returns the original
editor expression but with the current recursive term replaced by the
substitution. The type of the formal parameter of the function would depend on
where we are in the traversal, but the return type would always be an
\texttt{Edt}. This function could then be contained in the message giving us the following
message type.

\begin{lstlisting}[language=elm,%
                   gobble=4,%
                   ]
    type Msg a
        = SubstituteEdt (Edt a -> Edt a)
        | SubstituteAep (Aep a -> Edt a)
        | SubstituteAam (Aam a -> Edt a)
        | SubstituteEed (Eed a -> Edt a)
\end{lstlisting}

The \texttt{update} function can use the message variant to determine which type
of substitution is needed, and then call the function supplied with the message
to perform the substitution. This is the approach we went with since we do not
have to handle any error conditions --- like with the path approach --- as the
Elm type system helps us constrain the control flow to the valid code paths.

