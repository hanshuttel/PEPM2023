\section{The editor calculus}
\label{sec:editorcalculus}

\subsection{Syntax}

Editor expressions are given by the following formation rules.
%
\begin{align}
    \pi &::= \eval
  \mid \sub{D}
  \mid \child{n}
  \mid \parent \label{aep-formation-rules} \\
   \phi &::= \neg\phi
  \mid \conjunction{\phi_1}{\phi_2}
  \mid \disjunction{\phi_1}{\phi_2}
  \mid \at{D}
  \mid \possibly{D}
  \mid \necessarily{D} \label{eed-formation-rules} \\
    E &::= \pre{\pi}{E}
  \mid \bicond{\phi}{E_1}{E_2}
  \mid \seqcomp{E_1}{E_2}
  \mid \rec{x}{E}
  \mid \call{x}
  \mid \nil \label{edt-formation-rules} \\
    D & ::= \var{x}
  \mid \const{c}
  \mid \bind{\aamApp}{\tau_1\to\tau_2,\tau_1}  \label{eq:lan-mod-aam} \\
 &  \mid \bind{\aamLambda{x}}{\tau_1\to\tau_2} \mid \aamBreak  \mid \bind{\aamHole}{\tau}   \label{aam-formation-rules}
\end{align}
%
Editor calculus expressions $E$ can examine or modify the subtree of the
AST that is currently pointed to by the cursor. Expressions are built
from atomic edit action prefixes $\pi$. The \sub{D} prefix substitutes
the content under the cursor by the term constructor $D$.

Substitutions $D$ describe how the editor can
inspect and modify programs.
The interesting constructs are
$\bind{\aamHole}{\tau}$ which denotes a hole annotated with type
$\tau$, $\cursor{a}$ which denotes an expression that is directly
underneath the cursor and $\breakpoint{a}$, which is a breakpoint --
meaning that this occurrence of $a$ is left unevaluated.

A conditional expression \bicond{\phi}{E_1}{E_2} will proceed as $E_1$
if the condition $\phi$ holds for the current subtree and as $E_2$
otherwise. We allow for recursive expressions \rec{x}{E} where $x$
ranges over recursion variables (we only consider expressions where
every $x$ is bound by some \rec{x}{E}). Sequential composition
\seqcomp{E_1}{E_2} executes $E_1$ and, when this terminates, $E_2$ is
then execued.

Conditions $\phi$ are conditions from a spatial logic on ASTs that contains
the usual propositional connectives as well as the modalities
$\at{\phi}$, $\possibly{\phi}$ and $\necessarily{\phi}$. The formula
$\at{\phi}$ expresses that $\phi$ holds for the current subtree, while
$\possibly{\phi}$ expresses that $\phi$ in some subtree of the current
subtree and $\necessarily{\phi}$ expresses that $\phi$ holds
everywhere in the current subtree.

Finally, the $\eval$ primitive allows us to evaluate the entire AST
(up to possible breakpoints).

\subsection{Semantics}

\begin{figure*}
  \center
  \renewcommand{\arraystretch}{2}
  \begin{tabular}{llll}
    \scriptsize(COND-1)  & $ \CondOne $           & \scriptsize(COND-2) & $ \CondTwo$ \\
    \scriptsize(EVAL)    & $ \Eval $              & \scriptsize(SEQ)    & $ \Seq$     \\
    \scriptsize(STRUCT)  & $\Struct$              & \scriptsize(CONTEXT)& \scriptsize$\Context$
  \end{tabular}
  \caption{Examples of transition rules for editor expressions}
  \label{fig:edtreductionrules}
\end{figure*}

The semantics of the editor calculus is given by a small-step
operational semantics with transitions of the form
$\conf{E, a} \transition{$\alpha$} \conf{E', a'}$ where
$\alpha \in \mathcal{L}_E$ is an edit action or a value.

Editor expressions are always reduced in configurations of the form
$\conf{E,a}$, where $E$ is the given editor expression and $a$ is the
AST to which $E$ is applied. Figure
\ref{fig:edtreductionrules} shows some of the transition rules for editor
expressions.

% As an example, the rules \hyperref[fig:edtreductionrules]{(COND-1) and (COND-2)} describe that
% for a biconditional editor expression, $\bicond{\phi}{E_1}{E_2}$, we either use
% $E_1$ or $E_2$ as editor expression for the next reduction increment, based on
% whether or not $\phi$ holds for the AST in the configuration. Hence it just
% returns $\epsilon$.

\subsection{A type system for editor expressions}

The type system should guarantee that well-typed editor expressions
can only build well-typed programs. In order to achieve this, type
judgements are of the form
%
\[ p,\Gamma_e \vdash E : ok \]
%
Here $p$ is a tree path and $\Gamma_e$ is a type context that for any
given path $p_1$ gives us a pair $(\Gamma_a,\tau)$
where $\Gamma_a$ is a type environment such that $dom(\Gamma_a)$ is
the set of free program variables known at the end of the path $p_1$
and $\tau$ is the type of the subtree located at the end of the path.

Figure \ref{fig:edttyperules} shows a selection of the type rules.

\begin{figure*}
  \center
  \renewcommand{\arraystretch}{2}
  \begin{tabular}{llll}
    % \scriptsize(T-CH-1)      & $ \TChOne $             & \scriptsize(T-REF)       & $ \TRef$                \\
    % \scriptsize(T-CH-2)       & $ \TChTwo$               & \scriptsize(T-EVAL)      & $\TEval$               \\
    % \scriptsize(T-PARENT)      & $ \TParent$             & \scriptsize(T-NIL)    & $\TNil$              \\
%   \scriptsize(T-SEQ)       & \scriptsize$\TSeq$ & &\\
%    \scriptsize(T-REC)       & \scriptsize$\TRec$   & &   \\
    % \scriptsize(T-COND)      & \scriptsize$ \TCond$  & &\\
    % \scriptsize(T-SUB-VAR)   & $ \TSubVar$            &           & \\
    % \scriptsize(T-SUB-CONST) & $ \TSubConst$      & &\\
    \scriptsize(T-SUB-HOLE)  & $\TSubHole$ & &           \\[4mm]
%    \scriptsize(T-SUB-ABS)   & \scriptsize$\TSubAbs$   & & \\
    \scriptsize(T-SUB-APP)   & \scriptsize$ \TSubApp$ & & \\
%  \scriptsize(T-SUB-BREAK) & \scriptsize$\TSubBreak$ & &
  \end{tabular}
  \caption{Selected type rules for editor expressions}
  \label{fig:edttyperules}
\end{figure*}

%%% Local Variables:
%%% mode: latex
%%% TeX-master: "../pepm2023"
%%% End:
