% Created 2023-06-15 Thu 13:50
\documentclass[11pt]{article}
\usepackage[utf8]{inputenc}
\usepackage[T1]{fontenc}
\usepackage{fixltx2e}
\usepackage{graphicx}
\usepackage{longtable}
\usepackage{float}
\usepackage{wrapfig}
\usepackage{rotating}
\usepackage[normalem]{ulem}
\usepackage{amsmath}
\usepackage{textcomp}
\usepackage{marvosym}
\usepackage{wasysym}
\usepackage{amssymb}
\usepackage{hyperref}
\tolerance=1000
\author{Hans Hüttel}
\date{\today}
\title{paint-tanker}
\hypersetup{
  pdfkeywords={},
  pdfsubject={},
  pdfcreator={Emacs 25.3.50.1 (Org mode 8.2.10)}}
\begin{document}

\maketitle
\tableofcontents

\section{Om PAINT2023}
\label{sec-1}

\section{Skal passe med CFP; projectional editors er et fokusområde, så en lidt anden indgangsvinkel er nødvendig}
\label{sec-2}

\section{Hvad er fokus her?}
\label{sec-3}

\subsection{Selve implementationen og det at den er typet}
\label{sec-3-1}

\subsection{Hvad typesystemet understøtter her}
\label{sec-3-2}

\subsection{Udvidelse af editorkalkylen (jvf. PEPM 2021)}
\label{sec-3-3}

\subsection{Optage makroer [også et salgsargument! at man kan gemme refaktoriseringer]}
\label{sec-3-4}

\subsection{Mere om interface}
\label{sec-3-5}

\subsection{Et godt eksempel}
\label{sec-3-6}

\subsection{Video med eksempel (samme eksempel som i tekst)}
\label{sec-3-7}

\section{På denne måde: Udgangspunkt i det gamle PEPM2023-manuskript}
\label{sec-4}

\section{Michael Koldsgaard som medforfatter}
\label{sec-5}

\section{Link til kildekode: \url{https://gitlab.com/ntt-bachelor-project/editor}}
\label{sec-6}

Tidsplan for videre arbejde:

\section{Hvornår har vi tid?}
\label{sec-7}

\subsection{Thorbjørn har tid fra 24.6}
\label{sec-7-1}
\subsection{Hans har tid fra 29.6 (ferie fra 8.7)}
\label{sec-7-2}
\subsection{Michael har tid 5.7 og frem}
\label{sec-7-3}
\subsection{Nikolaj og Tórur har tid til mindre opgaver}
\label{sec-7-4}

\section{Arbejdsopgaver:}
\label{sec-8}

\section{24.6}
\label{sec-9}

\subsection{Thorbjørn har opdateret Github-koden (hvad er URL?)}
\label{sec-9-1}
\subsection{NTT: Finde et godt eksempel til artikel og video}
\label{sec-9-2}
\subsection{Thorbjørn og Michael: Finde den trælse fejl i implementationen}
\label{sec-9-3}

\section{29.6}
\label{sec-10}

\subsection{Hans: Opdatere indledning i artikel (og skære budskabet til ift. Call for Papers)}
\label{sec-10-1}
\subsection{NTT: Udvide implementation, så det er muligt at optage makroer}
\label{sec-10-2}

\section{3.7}
\label{sec-11}

\subsection{Zoom-møde, hvor vi gør status og planlægger sidste sprint [alle er inviteret]}
\label{sec-11-1}
\subsection{Optage kort video}
\label{sec-11-2}

\section{8.-9.7}
\label{sec-12}

\subsection{Hans: Sende ind til PAINT [så artiklen er de facto færdig til da]}
\label{sec-12-1}
% Emacs 25.3.50.1 (Org mode 8.2.10)
\end{document}