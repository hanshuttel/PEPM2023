\section{The editor calculus}

\subsection{Syntax}

Editor expressions are given by the following formation rules.
%
\begin{align}
    \pi &::= \eval
  \mid \sub{D}
  \mid \child{n}
  \mid \parent \label{aep-formation-rules} \\
   \phi &::= \neg\phi
  \mid \conjunction{\phi_1}{\phi_2}
  \mid \disjunction{\phi_1}{\phi_2}
  \mid \at{D}
  \mid \possibly{D}
  \mid \necessarily{D} \label{eed-formation-rules} \\
    E &::= \pre{\pi}{E}
  \mid \bicond{\phi}{E_1}{E_2}
  \mid \seqcomp{E_1}{E_2}
  \mid \rec{x}{E}
  \mid \call{x}
  \mid \nil \label{edt-formation-rules} \\
    D &::= \var{x}
  \mid \const{c}
  \mid \aamApp
  \mid \aamLambda{x}
  \mid \aamBreak
  \mid \aamHole \label{aam-formation-rules}
\end{align}

Editor calculus expressions $E$ can examine or modify the subtree of the
AST that is currently pointed to by the cursor. Expressions are built
from atomic edit action prefixes $\pi$. The \sub{D} prefix substitutes
the content under the cursor by the term constructor $D$. Moreover,
conditional expressions \bicond{\phi}{E_1}{E_2} will proceed as $E_1$
if the condition $\phi$ holds for the current subtree and as $E_2$
otherwise. We allow for recursive expressions \rec{x}{E} where $x$
ranges over recursion variables (we only consider expressions where
every $x$ is bound by some \rec{x}{E}) and sequential composition
\seqcomp{E_1}{E_2}.

Conditions $\phi$ are conditions from a spatial logic on ASTs that contains
the usual propositional connectives as well as the modalities
$\at{\phi}$, $\possibly{\phi}$ and $\necessarily{\phi}$. The formula
$\at{\phi}$ expresses that $\phi$ holds for the current subtree, while
$\possibly{\phi}$ expresses that $\phi$ in some subtree of the current
subtree and $\necessarily{\phi}$ expresses that $\phi$ holds
everywhere in the current subtree.

Finally, the $\eval$ primitive allows us to evaluate the entire AST
(up to possible breakpoints).

\subsection{Semantics}

\begin{figure*}
  \center
  \renewcommand{\arraystretch}{2}
  \begin{tabular}{llll}
    \scriptsize(COND-1)  & $ \CondOne $           & \scriptsize(COND-2) & $ \CondTwo$ \\
    \scriptsize(EVAL)    & $ \Eval $              & \scriptsize(SEQ)    & $ \Seq$     \\
    \scriptsize(STRUCT)  & $\Struct$              & \scriptsize(CONTEXT)& \scriptsize$\Context$
  \end{tabular}
  \caption{Editor Expression reduction rules}
  \label{fig:edtreductionrules}
\end{figure*}

The semantics of the editor calculus is given by a labelled transition system
$(S_E, \mathcal{L}_E, \to)$, where $S_E = \edt \times \AST$ and $\mathcal{L}_E
= \val \cup \aep \cup \{\epsilon\}$ and where transitions take the form
$\breakpoint{E, a} \transition{$\alpha$} \breakpoint{E', a'}$ with $a$ being
well-formed, $E$ being completed and $\alpha \in \mathcal{L}_E$. 

Editor expressions are always reduced in configurations of the form
$\breakpoint{E,a}$, where $E$ is the given editor expression and $a$ is the
currently build AST to which $E$ is applied. In terms of the labeled
transition system, we incrementally reduce the configuration until $E'$ becomes
$\nil$, and returns $\alpha$ for each increment. Figure
\ref{fig:edtreductionrules} show the different reduction rules for editor
expressions.

As an example, the rules \hyperref[fig:edtreductionrules]{(COND-1) and (COND-2)} describe that
for a biconditional editor expression, $\bicond{\phi}{E_1}{E_2}$, we either use
$E_1$ or $E_2$ as editor expression for the next reduction increment, based on
whether or not $\phi$ holds for the AST in the configuration. Hence it just
returns $\epsilon$.

\subsection{A type system}

%%% Local Variables:
%%% mode: latex
%%% TeX-master: "../pepm2023"
%%% End:
