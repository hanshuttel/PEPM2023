\section{Programs}

The program terms that editor expression terms can build and modify
are terms of a simply typed $\lambda$-calculus extended with
breakpoints $\breakpoint{a}$ and typed holes $\bind{\hole}{\tau}$.

The formation rules are
%
\begin{align}
  a &::= x
  \mid c
  \mid \app{a_1}{a_2}
  \mid \abs{\bind{x}{\tau}}{a}
  \mid \letin{\bind{x}{\tau_1}}{a_1}{a_2}\\
    &\mid \letrec{\bind{x}{\tau_1}}{a_1}{a_2}
  \mid l.a_1.a_2
  \mid l.a
  \mid \cursor{a}
  \mid \breakpoint{a}
  \mid \bind{\hole}{\tau}\\
  l &::= x_l
  \mid c_l
\end{align}
%
We use a big-step reduction semantics of our $\lambda$-calculus with
reductions of the form $a \to v$; it
will be used for defining the behaviour of the $\eval$ primitive in the
editor calculus. A selection of the rules can be seen in Figure \ref{fig:transitionrulesast}.

\begin{figure*}[]
  \center
  \begin{tabular}{llll}
    \scriptsize(B-APP)    & \scriptsize $\BApp$ &                              &              \\
    \scriptsize(B-APPB-1) & $ \BAppBOne $       & \scriptsize\text{(B-APPB-2)} & $ \BAppBTwo$ \\
    \scriptsize(B-APPH-1) & $ \BAppHOne $       & \scriptsize\text{(B-APPH-2)} & $ \BAppHTwo$
  \end{tabular}
  \caption{Selection of big-step reduction rules for programs}
  \label{fig:transitionrulesast}
\end{figure*}

Types $\tau$ are given by the formation rules
%
\[ \tau ::=  \tau_1 \rightarrow \tau_2 \mid b \]
%
where $b$ ranges over a set of base types.

Type judgements are of the form $\Gamma_a \vdash a : \tau$, where
$\Gamma_a$ is a type context that maps variables to types. The type
rules are given in Figure \ref{fig:asttyperules}.

\begin{figure*}
  \center
  \renewcommand{\arraystretch}{2}
  \begin{tabular}{llllll}
    \scriptsize(T-VAR)    & $ \TVar $   & \scriptsize(T-CONST) & $ \TConst$ & \scriptsize(T-CURSOR) & $ \TCursor $ \\
    \scriptsize(T-LAMBDA) & $ \TLambda$ & \scriptsize(T-BREAK) & $\TBreak$  & \scriptsize(T-HOLE)   & $\THole$     \\
    \scriptsize(T-APP)    & $\TApp$     &                      &            &                       &
  \end{tabular}
  \caption{AST type rules}
  \label{fig:asttyperules}
\end{figure*}

%%% Local Variables:
%%% mode: latex
%%% TeX-master: "../pepm2023"
%%% End:
