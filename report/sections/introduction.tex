\section{Introduction}
\label{introduction}

In their 2020 paper Godiksen et al. \pepm~described an editor calculus for a
typed structure editor that works directly with abstract syntax trees (ASTs)
making syntax errors impossible. Their editor calculus has structural
operational semantics and works for a simple lambda calculus that allows
evaluation of unfinished programs through the notion of holes and breakpoints.
The calculus uses editor expressions to modify ASTs and has bidirectional
communication between editor expressions and ASTs such that an editor
expression modifies an AST and an AST provides information that the editor
expression needs in order to reduce. Our project aims to work as a proof of
concept for the calculus.\\

There are a multitude of reasons as to why structure editors can work as an
alternative to the standard text editors. This includes the elimination of
syntax errors by working directly with ASTs and in other instances giving a
more intuitive visualization of the program.

An early example of such a structure editor is the Cornell Program Synthesizer
from 1981 \cornell. It directly edits the ASTs in programs by using a cursor to
select nodes and allows for insertions and modifications at the cursor. This
editor was not typed meaning that it allowed ill-typed ASTs to be built.

A later example is Hazelnut from 2017 \hazel. It is a bidirectionally typed
structure editor on ASTs, where it introduces holes that represent uncompleted
subtrees. It has dynamic semantics and type consistency of holes are not
checked until the hole is built to completeness. The result is that for
uncompleted programs the completed parts can still be evaluated and the holes
can still be meaningful, even when they are not well-typed. In comparison, the
type-safe editor calculus described by Godiksen et al. has static semantics and
does assign meaning to ill-typed holes, but still allows partial evaluation of
programs through the insertion of breakpoints.

The visualization of the Cornell Program Synthesizer and Hazelnut
implementations are both textual and close to that of a text editor. Some
structure editors that are motivated by unconventional (non text-based)
visualization include Scratch \scratch, where programming is done by moving
blocks of different color and combining blocks like a puzzle. Another inclusion
is Alice \alice, which is a programming language for writing 3D graphics where
the structure editor is visualized as 3D renderings. \\

In this paper we will introduce the implementation details of our structure
editor derived from the editor calculus described in the paper by Godiksen et
al. and argue for why our implementation retains all guarantees from the
calculus, even where the calculus may not be directly modeled.

We want to implement the editor in a functional programming language, as they
provide expressive types that lend themselves well to modeling the formal
descriptions of the editor. In particular, we want a typed language that has
algebraic data types. We considered using either Haskell or Elm, since both are
fairly popular functional programming languages. Haskell would be a good
choice, but we ended up choosing Elm, primarily because it would be easier to
create a user interface in Elm.\\

In order to create editor expressions that are used to traverse and modify the
ASTs, we introduce an editor expression builder that should allow us to
iteratively build editor expressions.

We want to visualize ASTs and editor expressions as well as interact with them.
By interacting we mean to be able to modify an AST by evaluating configurations
consisting of an AST and an editor expression, where the editor expression can
be any desired expression. Furthermore we want choices available in our visual
representation of ASTs, making it possible to change between different
visualizations.

Godiksen et al. also describe a way to acquire type-safety in their calculus,
by way of their type system. The implementation of their type system should
make it impossible to get runtime errors from problems such as invalid ASTs as
well as expressions acting upon non-existing subtrees et cetera.

Going from a completely theoretical calculus to a practical implementation, we
also have to consider the usability of the editor. While not the main focus we
will go through approaches where we make it easier to use certain aspects of
the editor, such as using editor expressions and navigating the ASTs. \\

This paper is structured the following way. Preliminaries are listed in section
\ref{preliminaries}. The overall principles of Godiksen et al.s editor calculus
are described in section \ref{principles}. In section \ref{design} we introduce
the design of the implementation. Section \ref{modeling} is about how we model
the calculus terms, types and the editor expression builder. The evaluation
rules are defined in section \ref{evaluation} together with implementation and
execution of said rules. In section \ref{type-checking} we define the type
rules and describe the implementation of the type checker. We discuss the user
interface and how it combines the calculus in section \ref{user-interface}. In
the last section - section \ref{conclusion} - we conclude our project and
discuss future work.

