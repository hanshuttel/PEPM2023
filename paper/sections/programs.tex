\section{Programs}

The program terms that editor expression terms can build and modify
are terms of a simply typed $\lambda$-calculus extended with
breakpoints and holes. The formation rules are

\begin{align}
  D & ::= \var{x}
  \mid \const{c}
  \mid \bind{\aamApp}{\tau_1\to\tau_2,\tau_1}  \\ \label{eq:lan-mod-aam}
&  \mid \bind{\aamLambda{x}}{\tau_1\to\tau_2} 
  \mid \aamBreak
  \mid \bind{\aamHole}{\tau} \\ 
  a & ::= x
  \mid c
  \mid \app{a_1}{a_2}
  \mid \abs{\bind{x}{\tau}}{a}
  \mid \cursor{a}
  \mid \breakpoint{a}
  \mid \bind{\hole}{\tau} \label{eq:lan-mod-ast}
\end{align}

Terms are extended with constructs that describe how the editor can
inspect and modify programs. The new constructs are
$\bind{\aamHole}{\tau}$ which denotes a hole annotated with type
$\tau$, $\cursor{a}$ which denotes an expression that is directly
underneath the cursor and $\breakpoint{a}$, which is a breakpoint --
meaning that this occurrence of $a$ is left unevaluated.

Types $\tau$ are given by the formation rules
%
\[ \tau ::=  \tau_1 \rightarrow \tau_:2 \mid b \]
%
where $b$ ranges over a set of base types.

Type judgements are of the form $\Gamma_a \vdash a : \tau$, where
$\Gamma_a$ is a type context. The type rules are given in Figure
\ref{fig:asttyperules}.

\begin{figure*}
  \center
  \renewcommand{\arraystretch}{2}
  \begin{tabular}{llllll}
    \scriptsize(T-VAR)    & $ \TVar $   & \scriptsize(T-CONST) & $ \TConst$ & \scriptsize(T-CURSOR) & $ \TCursor $ \\
    \scriptsize(T-LAMBDA) & $ \TLambda$ & \scriptsize(T-BREAK) & $\TBreak$  & \scriptsize(T-HOLE)   & $\THole$     \\
    \scriptsize(T-APP)    & $\TApp$     &                      &            &                       &
  \end{tabular}
  \caption{AST type rules}
  \label{fig:asttyperules}
\end{figure*}

%%% Local Variables:
%%% mode: latex
%%% TeX-master: "../pepm2023"
%%% End:
