\section{Introduction}
\label{introduction}

Structure editors, also known as projectional editors, are an
alternative to the standard text editors that eliminate the occurrence
of syntax errors by working directly with an abstract syntax tree
(AST) and can also be used to give a more intuitive visualization of
the program code.

An early example of a structure editor is the Cornell Program Synthesizer
from 1981 \cornell. It directly edits the ASTs in programs by using a cursor to
select nodes and allows for insertions and modifications at the cursor. 

A more recent example is Hazel \hazel. This is a structure editor on
ASTs which introduces holes that represent uncompleted subtrees. The
language of edit actions in Hazel is described in the Hazelnut editor
calculus which comes with a reduction semantics and a bidirectional
type system. The calculus has a dynamic semantics and type consistency
of holes is not checked until the hole is built to completeness. The
result is that for uncompleted programs the completed parts can still
be evaluated and the holes can still be meaningful, even when they are
not well-typed.

In their 2020 paper Godiksen et al. \pepm~describe an editor calculus
inspired by Hazelnut. The calculus describes the edit actions of a
typed structure editor on terms of a simply-typed lambda calculus. The
calculus allows for partial evaluation of terms by means of
breakpoints and like Hazelnut, terms can be incomplete in the form of
holes. However, unlike Hazelnut which only allows for finite edit
expressions, this calculus incorporates recursive behaviours and a
very expressive spatial logic on ASTs. The type system of the calculus 
guarantees that well-typed editor expressions will always construct
well-typed programs.

In this paper we extend the programming language and editor calculus
of \pepm such that we now obtain a full programming language and show
that the editor calculus can serve as the basis of a structure editor
for our language by means of an implementation in Elm. This
implementation effort involves an implementation of the spatial logic
as well as the underlying type system.

Moreover, in order to create editor expressions that are used to
traverse and modify the ASTs, we introduce an editor expression
builder that allows us to iteratively build editor expressions and a
macro recorder that lets us construct editor expressions from single
edits actions.

The syntax, semantics and type system of the editor calculus and our
programming language are described in section
\ref{sec:editorcalculus}. In section \ref{sec:implementing} we outline
the design of the implementation.


%%% Local Variables:
%%% mode: latex
%%% TeX-master: "../pepm2023"
%%% End:
