\section{Principles of the editor calculus}
\label{principles}

The editor calculus defined by Godiksen et al. can iteratively build, modify
and evaluate ASTs with editor expressions through bidirectional communication in
a type-safe manner \pepm. ASTs are denoted $a$ and their syntax is formed by
seven different terms, the most simple being that of variables, constants,
lambdas and applications. Then we also have cursors, which are used to indicate
where we are editing the structure of an AST. We say an AST is well-formed iff
it has one cursor. Lastly we have breakpoints and holes. Breakpoints allow for
partial evaluation of ASTs i.e. subtrees beneath breakpoints are not evaluated,
and holes are unfinished subtrees.

Editor expressions are denoted $E$ and are chains of different commands and
conditions. The trivial case that end editor expressions is the empty editor
expression, denoted $\nil$. Then we have prefix commands, biconditional
expressions, sequential composition and recursive expressions. The prefix
commands can tell to evaluate, substitute at cursor position or move cursor
position in an AST. We describe the formation of ASTs and editor expressions
further in section \ref{modeling}.

The semantics is bidirectional communication where the editor expression
reduction is relative to some AST, and the AST provides information for the
reduction. Thus the semantics is defined relative to configurations of an AST
and an editor expression, denoted $\breakpoint{E,a}$. When editor expressions
are reduced we get new expressions and ASTs and it is defined in a labeled
transition system where transitions take the form $\breakpoint{E, a}
\transition{$\alpha$} \breakpoint{E', a'}$. The calculus also defines labeled
transition systems for reducing prefix commands and for evaluating ASTs. The
reduction of conditional editor expressions are defined in a modal logic system
with negation, conjunction, disjunction and three modalities. We will go into
more details about reduction semantics, reduction rules and systems in section
\ref{evaluation}.

In their paper, Godiksen et al. define a sound type system for the calculus
\pepm. In the editor we judge the well-typedness of ASTs and editor expressions
relative to either an AST type context or an editor expression type context and
a path to the cursor, respectively, which we denote $\ctx{a} \vdash
\bind{a}{\tau}$ and $p,\ctx{e} \vdash \bind{E}{ok}$, respectively. We type
check configurations before we resolve them. A configuration is well-typed when
both the AST is well-typed in the empty AST type context and the editor
expression is well-typed in the path to the cursor and the editor type context
of the AST. In section \ref{modeling} and \ref{type-checking} we will define
and outline the detail of paths, contexts and the type rules.
